\begin{figure*}
  \centering
\includegraphics[width=.75\linewidth]{fig/system/system.pdf}
\caption{\label{fig:system}A diagram of RCloud's architecture. }
\end{figure*}

\section{Related Work\label{sec:related}}

In some sense or another, the entire field of information
visualization and visual analytics revolve around improving how users
solve problems using data. In this section, we focus specifically on
system work relating to problem-solving environments and multiple users.

\paragraph*{Social Data exploration and Analysis}
{\bf ManyEyes}~\cite{Viegas:2007:MAS} was a landmark system designed
for the crowdsourced creation and publication of data
visualizations. Although ManyEyes only supported a limited number of
visual encodings, the system's success was both a precursor to more
sophisticated solutions and an early indication that the world-wide
web was a suitable platform for data visualization. One of the main
challenges we faced in designing RCloud was providing an experience
for \emph{consuming} visualizations as seamless as ManyEyes's, while
sacrificing as little as possible on the generality of the analyses
themselves.

\paragraph*{Notebooks as a medium for data analysis dissemination}
The concept of a ``notebook'' as we use it here can be traced all the
way back to Knuth's literate programming~\cite{Knuth:1984:LP}. In
literate programming, a comprehensive description in prose of the
behavior of the program is ``weaved'' together with the source code,
yielding both an executable program and a human-readable document.  A
notebook represented as a collection of short, executable cells,
originated with Mathematica, and in R, literate programming is
supported by packages such as knitr and RMarkdown~\cite{Xie:2013:DDW}.
Project Jupyter~\cite{jupyter} (originally implemented as part of
IPython) offer some notebook features, but lack a transparent
mechanism for sharing and deployment in multiple-user settings.
%
Further afield, {\bf Electronic Lab Notebooks} are applications for organizing
and sharing data from scientific lab experiments\cite{Rubacha:2011:ELN}.
In a sense, we hope to adapt and extend this concept to the work of
visual analytics teams.
%
Although RStudio offers publication of literate R programs as a free
service on their website, the workflow is somewhat disconnected from
the development of those programs. Once they're uploaded, it's hard
for other users to build off of the work published (or even for the
original author to update new versions). In other words, RStudio
handles \emph{publication}, but not \emph{deployment} and sharing.

\paragraph*{Provenance and versioning} As
we mentioned, one central problem in exploratory analysis is that
problems change quickly over time, often in the course of developing a
solution. As a result, systems should provide adequate support for
tracking \emph{changes} of the data analyses scripts.  VisTrails was
one of the original systems for managing \emph{process provenance} ,
and demonstrated the value of capturing aspects of the processes that
surround data analysis experiments and tools, including detailed
history, collaboration, and deployment~\cite{Callahan:2006:VVM}.

VisMashup~\cite{Santos:2009:VST}  defines a schema and
semantics for automatically deriving user interfaces from workflows,
while Crowdlabs exposes these capabilities on a website
feature~\cite{Mates:2011:CSA} workflow upload and remote execution. In
our view, the impedance mismatch between a dataflow pipeline
specification and the power of a general-purpose language is too great
for the type of general exploratory work in data science teams. At the
expense of ease of use for non-programmers, RCloud tries to provide a
closer match for analysts accustomed to creating and executing R and
Python code, while retaining attractive properties like transparent
provenance tracking and interactive data visualization on the web.

\paragraph*{Web-based tools for sharing code snippets}
There have been a variety of tools recently developed for quickly
sharing small programs on the Web, including but not limited to:
bl.ocks~\cite{blocks}, jsfiddle~\cite{jsfiddle}, and
plot.ly~\cite{plotly}.  bl.ocks and jsfiddle are both designed to
share Javascript programs, which means that their deployment happens
automatically through the web browser. If and when Javascript becomes
the lingua franca of exploratory data analysis, then we can foresee
building a much simpler version of RCloud where all execution happens
either on the client side or via web services. Unfortunately that is
not a realistic assumption for the present, making these solutions by
themselves unsuitable to our use case. Plot.ly is notable in that it
provides API support for publishing \emph{from} scripts: in other
words, it is possible to generate a plot.ly visualization from inside
a separate program. Although this is an intriguing idea, it
nevertheless creates a disconnect between the analysis and the
resulting visualization. In RCloud, we wanted to make sure that every
visualization was transparently linked to the source code that
generated it.

%% publishing web content such as knitR, Rpubs,
%% rCharts, and gg2v.  These tools are not collaborative, but they are
%% aimed at better graphics and web applications.
%% .  {\bf iPython
%%   notebooks (Jupyter)} are the closest work we are aware of.  The goal
%% of this project is to provide shared notebooks and a remote execution
%% environment for interpreted scripts.  \stephen{Are there significant
%%   differences involving versioning, provenance, annotation and other
%%   broad process support?}


%% {\bf bl.ocks, jsfiddle, plotly} and other web services for sharing code
%% and demos.


\paragraph*{Needs of data analysts}
Kandel et al.'s interview study points out the typical ``explore'',
``model'', ``report'' cycle in enterprise data
analysis~\cite{Kandel:2012:EDA}. There are many discontinuities in
this cycle that cost time and effort to overcome. RCloud seeks to
reduce this mismatch. Kandel et al also point out that larger teams
are becoming more common in data analysis, that supporting
collaboration is difficult and important, and that sharing
and versioning of data sources and artifacts is hindered by current
technology. ``We found that analysts typically did not
share scripts with each other. Scripts that were shared were
disseminated similarly to intermediate data: either through shared
drives or email. Analysts rarely stored their analytic code in source
control.'' Their study highlights the opportunity for better ways
of supporting collaboration and sharing in data analysis teams.

An earlier study by Kandel et al argues that data wrangling
(cleaning, parsing and transformation)is a major part of exploratory
analysis and visualization~\cite{Kandel:2011:RDI}. We view this
as attacking a different semantic level than ours, but also
showing the need for an environment that enables better sharing
of the knowledge, tools and processes to do this. Anecdotally
we find much frustration among practitioners that this knowledge
is difficult to find and often is not recorded or available in a
reusable form even within the same organization.

Heer and Agarwala identify many design considerations for
collaborative visual analytics~\cite{Heer:2008:DCF}.
RCloud notebooks, and the integrated version control system for them,
described in Section~\ref{sec:notebooks}, address modularity and granularity,
and artifact histories.
\emph{Starring}, the means for signaling interest in notebooks, described in
Section~\ref{sec:starring}, addresses social-psychological incentives,
recommendation, and voting and ranking. RCloud's integrated deployment
mechanism, described in Section~\ref{sec:deployment}, addresses the cost of
integration, content export, presentation and view sharing.

The need for integrating statistics and visualization has been
highlighted in previous studies and is widely understood by
various technical communities \cite{Perer:2008:ISA}
Lucas and Roth were early advocates of combining
data exploration with presentation and publication \cite{Lucas:1996:EIV}.

There has been noteworthy work on specific techniques such as
social bookmarking \cite{Millen:2006:DSB} \cite{Heer:2007:VAV}
and crowdsourcing \cite{Fast:2014:ECS} to support collaborative
or social development or analysis processes.
Similarly, there are computational methods to support high
performance execution in incremental code development
environments \cite{Guo:2010:TPI}.
The goal of RCloud is to define an environment in which many such
techniques may be integrated and made available to a broad community.
\begin{figure*}
\centering
\includegraphics[width=.95\linewidth]{fig/notebook/notebook.pdf}
\caption{\label{fig:notebook}An RCloud notebook is a sequence of
  \emph{cells}, each a snippet of source in one of the supported languages (typically R, but Python and others are also supported) or Markdown. The main creation workflow involves editing notebooks, which are transparently stored as git repositories in GitHub, providing us with easy access primitives for version tracking. Notebooks can be executed as they're edited (left), or in a standalone viewer (right), via a slightly different URL. This lightweight provides a very low-friction mechanism for sharing results which we discuss in full in Section~\ref{sec:interviews}. }
\end{figure*}

%% \stephen{We have a problem we already went over some of this.}
%% Our work has been inspired by and benefited from other efforts
%% to improve data analysis environments and processes. Building in
%% the immense foundation of systems for statistical graphics,
%% recent work includes RStudio \cite{RStudio:2013:SWA},
%% the R packages Markdown \cite{Allaire:2014:MMR},
%% knitr \cite{Xie:2013:DDW}
%% and Shiny \cite{RStudio:2013:SWA},
%% and iPython notebooks \cite{Perez:2007:IAS}
%% to name a few. RStudio aims at providing an integrated development environment
%% for R, with support for publishing code in packages. Markdown,
%% knitR and Shiny augment R with sophisticated reporting capabilities, including
%% interactive web interfaces. IPython \cite{Perez:2007:IAS}
%% shares many of our goals, such as providing a comprehensive environment
%% for analysis and programming, and share-able documents on the web.
%% \stephen{Do we now need to introduce one of those remarks where we
%% explain that we are not the same as IPython.}

One overall goal is to reduce the gap between implementers and
deployers in visual analytics. The fusion of development with
production operations in software release management (``DevOps''
\cite{Httermann:2012:DD} or ``continuous integration''
\cite{Fowler:2006:Continuous}) is a trend in web services and related
fields.  By making it convenient for data scientists to expend just a
little more effort when creating experiments, we may be able to
eliminate the need for programming teams to recreate their work to
deploy it in production, which has a high cost in time, expense and
accuracy. We now turn to a description of the system architecture
itself, and how it enables the features we have been highlighting.
