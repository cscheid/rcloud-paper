\section{Evaluation\label{sec:interviews}}

To evaluate the effectiveness of RCloud, we informally interviewed 13 current or
recent users of RCloud. 9 are data analysts, and 4 build tools for data
analysts and business needs. Users felt that sharing works well, but the notebook
interface slows down exploration, and the browsing interface is not scaling well.
%gw (trying to be specific)

\paragraph*{Sharing of results} is the core feature of RCloud, and a popular one. All
the subjects praised this feature.

By default, all RCloud notebooks are publicly visible, and notebooks can be
found by navigating the notebook tree, or by searching. However, users most
often mentioned sharing by sending links through email.

Besides providing a way to present work, notebook sharing become a starting point
in coding. Lilo says, ``The best part is how easily you can
share code. You can find a working example, rather than wearing
out Google and finding questionable examples that may or may not work.'' Wendy
notes, ``[If] some person has done something similar, then you're able to just
edit that, and that's saved a lot of work for me.''

Eric develops packages for analysts, and uses RCloud ``for sharing code with
other people, and for doing tutorials for {\em iotools} or {\em hmr}'', his packages.
``I want people to see how the package works, so I clearly want them
to see the code... I write it just like I would write GitHub
Markdown, where you have little code snippets and text, but RCloud lets me
actually run the snippets [and display the results].'' He also uses RCloud for
describing and debugging data sources.

Some users, who tried RCloud and were not able to continue for organizational
reasons, miss certain capabilities. Leith ``[likes] \comment{the concept of} being
able to create notebooks and share them... A wiki is not the best way for
communicating results - it's \comment{kind of} like writing a blog post with very limited
functionality. I have to save every picture and post it as an image.'' He adds,
``I can't share all of the code because it would just get crowded and wouldn't
look right on a wiki.'' Iris explains, ``If I could make a folder on RCloud and
have Python notebooks and also Pig notebooks there, and execute them from RCloud,
that would be much better than my current [environment], because that would free
me from manual documentation and version control and also telling people where
my code was. It would be just, hey, go look on RCloud, here's the stuff.''

\paragraph*{Forking} The ability to fork someone's notebook and continue their
work proves to be a useful feature, but it is a lousy way to change simple parameters.
Almost twice as many (131) users have forked
someone else's notebook as have starred one (75). Lilo says, ``It's one of the
handiest functions, because instead of having to find it, copy, paste it, you
just hit Fork, rename it, and it's done. It's pretty amazing.''

Although we intended forking to be available to improve others' code, we
didn't anticipate users forking their own notebooks, which
proved very useful. Kenyon says, ``I fork my own notebooks because I'm going off
and doing some other analogous project, so I've got interesting content that I've
already done in a previous analysis, that I want to start from and then tweak
to match a new set of data.''

Forking also provides a way for others to troubleshoot when something goes
wrong. When Hugh collaborates with users of his notebooks, ``I'll teach people
to intercept the result in the middle, to insert print statements here and
there and check values.''

A common but ill-advised use of forking is to change
parameters. Eric complains that a notebook might say ```This is a report of
the volume of all of our feeds for this month', and someone would want to
look at it for the next month or the previous month, so they'd fork it to
change the month.''

Parameters can be added to a notebook URL, but adding user interface
elements to do the same thing takes more expertise. Tool builders told us it
should be easier. Allison explains: ``[Users] can always fork the notebooks
and make changes, but I
feel that if the owner of the notebook [adds a user interface], it's easier to
run. Instead of forking it, if they can set options, it's probably more
efficient, and they can [still] fork it if they want to.''

\paragraph*{Automatic source control} is appreciated for its safety and ease,
but the large number of commits can be hard to manage. Lilo says, ``Instead of
looking back and saying I've got a billion files here in this subdirectory and I
hope I've got them backed up, if they're on RCloud I know they are.''

Iris points out that automatic versioning works well for coping with
the details of web development:
\begin{quote}
I like the fact that it has a built-in editor, so
if you need to fix a typo in a link, or an extra line break or other
nonsense, you can just switch to the edit view, pull up your asset, type
something, it's automatically saved, committed, everything. You don't have to go
back to your source code, change it, commit it to the repo, pull the repo to
your distribution version.
\end{quote}

On the other hand, saving every change leads to many fine-grained
versions. Kenyon says that for this reason, the history feature is not that
helpful: ``I don't need something that keeps track of every mistake I've made or
every direction I've tried.''

\paragraph*{Discovering others' work} Users reported that the search function is
more helpful for learning about
particular functions than techniques. Wendy says, ``The fact that we have
all the notebooks there, searchable, saves me from replicating what other people
have done.''

But other users prefer to browse just the notebooks of experts they
know. Kenyon says, ``Usually I know somebody's notebooks that I want to search
through, because \comment{I know that} the kind of thing I'm looking for is something
more obscure than I'm likely to find in some random person's.''
More selective ways to search will be needed as the number of notebooks
grows further. We mention possible approaches in Section~\ref{sec:discussion}.

\paragraph*{Integrated analysis} Integrating an analysis language into a web
development environment is something tool developers really appreciate.
The structure of Hugh's visualization notebook means
\begin{quote}
the other guys who want to do analytics on the data can first pull
the data, do the analytics on it, and then feed the viewer the data.
Anyone from the stats group can insert something in between.
Once you get the data into RCloud, then you have a dataframe
to work with, and then they can produce another dataframe.
\end{quote}

Integrating R makes Allison's application development easier:
``Having an open session where I can run R commands or
functions without having to invoke an API or send a request and then
wait for the response is extremely helpful in writing the application.''

\paragraph*{Web is limiting} Although most analysts appreciated RCloud's sharing features,
it was not a popular tool for exploratory data analysis.
All analysts with R experience prefeered to do analysis in another
tool, then paste code into RCloud for sharing, because the web interface got in the
way of their work.

The web interface of RCloud isn't satisfactory to Kenyon: ``It's nice to have
it saved, but there's this trade-off between it making it easier for me to
present something or to save something, and my ease of typing and
correcting and things in a plain editor window.''

Coby also works with text files and commands, rearranging source files
so active code is at the top. RCloud does not readily support this
workflow, so Coby often shares work by pasting code into an RCloud
notebook after he's done.

Working in a shared environment also entails compromises about what you can
install. Joy says installing alternative or nonstandard packages is
intrinsic to exploratory data analysis. This is a weakness in shared
environments like RCloud.

\paragraph*{Impermeable cells} RCloud's notebook interface combines editable
Markdown with a command line
interface. Many users felt that this format, so effective for presentation,
is incompatible with exploratory data analysis.

Much of the time, commands typed at the R command line serve only a transitory
purpose, so having RCloud persist commands in cells can be annoying.
Coby notes, ``It saves everything I do like everything is gold, but most
of it is junk not meant to be saved.''

Material that is not appropriate to save includes ``expressions that allow
me to check that I'm the right track'' (Kenyon), ``checking out what your data is,
or you make a plot of the data. Things that should not really become part of a
notebook, but things that help you understand your data better'' (Wendy).

Users felt that cells do not capture the right level of granularity.
Kenyon says that RCloud's cell structure
``tempts me to type a big long thing and then run the whole thing, as opposed to
typing a few little pieces and then put them together'' as he would on the command line.
When Eric uses the R command line, he ``copies and pastes 5-10 lines of code,
so when something breaks, I get an error message on that one line, and I can
up-arrow and change it and fix it, whereas in RCloud I have to run a whole cell,
so the only way to get that functionality is if every line's in one cell.''

\paragraph*{Too vertical} The vertical orientation of a notebook also poses problems
not encountered on the command line, where typically only the last few page of output matters.

Scrolling between code and results is time consuming, and some users would
prefer to keep results separate from code.
Gerrard thinks RStudio's layout is more helpful because charts
are shown in another pane that stays in place while doing analysis. Wendy says
``Cells are really useful, [but] you want to see your output in a different
window or on a separate part of the window''.

The cell structure also can be problematic if some cells take a long time to
execute. High variance is common: cells may take anywhere from seconds to
hours to run. In this situation, the Run Whole Notebook button is dangerous.
Coby reports that when converting his work to notebooks, he
ends up with a lot of comments saying ``This cell takes a long time to run.''
To avoid this pitfall, some users write code to explicitly cache results.

\paragraph*{A sea of notebooks} RCloud is becoming a victim of its own success, as it is becoming
difficult to navigate all the notebooks. There are over 5200 notebooks in
the research instance, of which more than 500 have been starred, and more
than 350 have been forked.

For this reason, Kenyon doesn't find the notebook tree satisfactory:
\begin{quote}
I don't necessarily need to see everybody's notebook that uses RCloud. Every
time I do something new, I get a new notebook, so now I have 50 or 60 notebooks.
That's enough to think about just on my own, but if everybody has
50 or 60 sitting on my display, it's more than I want to know about.
\end{quote}

Although RCloud promises an environment where notebooks should keep working,
not all our users have learned the habits that make this a reality. As Joy
puts it, ``bitrot'' is still a big problem -- notebooks often
stop working because the user changed the structure of their data, or
changed a filename or a database. Even if one forks someone
else's notebook and corrects it, the original notebook still exists with
the error. She says we need ``organizational protocols'' to catch up with the technology.

Eric, who writes packages and example notebooks for them, mentioned the
accumulation of dead notebooks. When Eric and Hugh work together on a notebook:
\begin{quote}
There's no way for both of us to have ownership of a notebook, so the only
way is to fork it back and forth, and so we have dozens of old copies. We
end up deleting all the old ones, but people still have links to them,
because they don't actually disappear.
\end{quote}

Eric tried to keep his notebook
tree well organized, but this didn't help, because people kept old links
in email, or forked notebooks which had become obsolete.
He says he became scared to share notebooks: ``Do I want to support this forever?''
