% part1.txt
Erin Shellman [Nordstrom]: Finally, prototyping our products so that internal
customers can use them early on has been crucial for
our success. It doesn't even have to be something
fancy. For instance, our recommendations preview
tool doesn't have a particularly interesting
visualization, but its enough to show the result of our
algorithms. Now we can shoot off a URL to internal
customers and it allows them to sit at their desk and
understand the behavior of our product and
experiment with it, and provide feedback way before
we're talking about getting it into production. This
has been super helpful and has been a really great
way to get people excited about what we're working
on. Building and maintaining those relationships are
just like anything else-- you always have to be
working on them. Nurturing those relationships are
part of the job, and if you leave them unattended,
they might not be there later.

% part2.txt
Gutierrez: How do you share the knowledge you are
building with others?
Shellman: We’re obsessed with Confluence from
Atlassian, and the Data Lab has a very active
Confluence space. Recommendo is fully documented
on Confluence, so anyone in the company could learn
how it’s built, where to find it, and how to use it. We
also share exploratory analyses and reports on
Confluence so that we can still exchange knowledge
even if the work didn’t make it into a larger project.

% part3.txt
John Foreman: I lead the data science team at
MailChimp, and I like to get my hands dirty too.
Some of the big pieces of my day involve working
with my team to take stock of current projects and
figure out where to go next, doing my own work and
prototyping things-- I do projects just like my peers
do and then also my talking with other teams,
talking with management, and planning for the
future.
On our team, we’ve got different folks facing different
% part4.txt
kinds of projects. We’ve got one person who really
owns compliance and looks at our compliance
processes. We’ve got another data scientist who
focuses on more of the user experience side of the
house and understanding our customers. I help them
and others as is needed while I do the three things
mentioned earlier-- build data products, be a
translator, and have conversations with the data
science community.

% part5.txt
Jonathan Lenaghan [PlaceIQ]: First and foremost, it is very important
to be self-critical: always question your assumptions
and be paranoid about your outputs. That is the easy
part, In terms of skills that people should have if they
really want to succeed in the data science field, it is
essential to have good software engineering skills. So
even though we may hire people who come in with
very little programming experience, we work very
hard to instill in them very quickly the importance of
engineering, engineering practices, and a lot of good
agile programming practices, This is helpful to them
and us, as these can all be applied almost one-to-one
to data science right now.
If you look at dev ops right now, they have things
such as continuous integration, continuous build,
automated testing, and test harnesses-- a1l of which
map very well from the dev ops world to the data ops
(a phrase I stole from Red Monk) world very easily. I
think this is a very powerful notion. It is important to
have testing frameworks for all of your data, so that if
you make a code change, you can go back and test all
of your data. % Having an engineering mindset is


% part6.txt
Anna Smith: In addition to Bitly's extraordinary data sets,
I was given all the resources that I needed. These were
resources that I’d only heard about but had never
seen. I went from asking what a Hadoop cluster was,
as my school didn't have one, to being able to work
with one. Now I could play with one to figure out how
to make my work better and how to change my
algorithms so they were cleaner and ran faster on
Hadoop. The whole transition was eye opening. Not
only that, I could also look at other people’s code and
play with their code and data sets as well. It was
fantastic and I was convinced that working with data
was my thing.
