%\documentclass[journal]{vgtc}                % final (journal style)
\documentclass[review,journal]{vgtc}         % review (journal style)
%\documentclass[widereview]{vgtc}             % wide-spaced review
%\documentclass[preprint,journal]{vgtc}       % preprint (journal style)
%\documentclass[electronic,journal]{vgtc}     % electronic version, journal

%% Uncomment one of the lines above depending on where your paper is
%% in the conference process. ``review'' and ``widereview'' are for review
%% submission, ``preprint'' is for pre-publication, and the final version
%% doesn't use a specific qualifier. Further, ``electronic'' includes
%% hyperreferences for more convenient online viewing.

%% Please use one of the ``review'' options in combination with the
%% assigned online id (see below) ONLY if your paper uses a double blind
%% review process. Some conferences, like IEEE Vis and InfoVis, have NOT
%% in the past.

%% Please note that the use of figures other than the optional teaser is not permitted on the first page
%% of the journal version.  Figures should begin on the second page and be
%% in CMYK or Grey scale format, otherwise, colour shifting may occur
%% during the printing process.  Papers submitted with figures other than the optional teaser on the
%% first page will be refused.

%% These three lines bring in essential packages: ``mathptmx'' for Type 1
%% typefaces, ``graphicx'' for inclusion of EPS figures. and ``times''
%% for proper handling of the times font family.

\usepackage{mathptmx}
\usepackage{graphicx}
\usepackage{times}
\usepackage{xspace}
\usepackage{amssymb}

% \usepackage[onecolumn]{multicol}

%% algorithm packages (llins)
\usepackage{algorithm}
\usepackage{microtype}
\usepackage{algpseudocode}

%% We encourage the use of mathptmx for consistent usage of times font
%% throughout the proceedings. However, if you encounter conflicts
%% with other math-related packages, you may want to disable it.

%% This turns references into clickable hyperlinks.
\usepackage[bookmarks,backref=true,linkcolor=black]{hyperref} %,colorlinks
\hypersetup{
  pdfauthor = {},
  pdftitle = {},
  pdfsubject = {},
  pdfkeywords = {},
  colorlinks=true,
  linkcolor= black,
  citecolor= black,
  pageanchor=true,
  urlcolor = black,
  plainpages = false,
  linktocpage
}

%% If you are submitting a paper to a conference for review with a double
%% blind reviewing process, please replace the value ``0'' below with your
%% OnlineID. Otherwise, you may safely leave it at ``0''.
\onlineid{}

%% declare the category of your paper, only shown in review mode
\vgtccategory{Research}

%% allow for this line if you want the electronic option to work properly
\vgtcinsertpkg

%% In preprint mode you may define your own headline.
%\preprinttext{To appear in an IEEE VGTC sponsored conference.}

%% Paper title.

\title{RCloud: Integrating Exploratory Visualization, Analysis and Deployment}

%% This is how authors are specified in the journal style

%% indicate IEEE Member or Student Member in form indicated below
\author{Stephen North and Carlos Scheidegger and Simon Urbanek and
  Gordon Woodhull}

\authorfooter{
%% insert punctuation at end of each item
}

%other entries to be set up for journal
% \shortauthortitle{Biv \MakeLowercase{\textit{et al.}}: Global Illumination for Fun and Profit}
%\shortauthortitle{Firstauthor \MakeLowercase{\textit{et al.}}: Paper Title}

%% Abstract section.
\abstract{ Consider the emerging role of a data science team within an
  organization. Individual data scientists and statisticians usually
  work on loosely related problems, and must devise effective ways to 
  share their findings and to move results from exploratory data analysis
  to automated diagnostics and reports deployed for wider consumption.
  There are two problems with the current
  practice. First, there are gaps in this workflow: EDA is performed
  with one set of tools, and automated reports and deployments with another.
  Second, EDA environments often assume a single-developer perspective,
  while data scientist teams could greatly benefit from easier sharing of
  scripts and data feeds, experiments, annotations, and automated
  recommendations, which are well beyond what traditional version control
  systems provide.  In this paper, we propose RCloud, a system that supports
  collaborative data analysis, visualization and web deployment. We will
  discuss the design decisions, tradeoffs and limitations, and compare RCloud
  to other current proposals.
} % end of abstract

%% Keywords that describe your work. Will show as 'Index Terms' in journal
%% please capitalize first letter and insert punctuation after last keyword
\keywords{Some, Keywords, Here}

%% \CCScatlist{ % not used in journal version
%%  \CCScat{K.6.1}{Management of Computing and Information Systems}%
%% {Project and People Management}{Life Cycle};
%%  \CCScat{K.7.m}{The Computing Profession}{Miscellaneous}{Ethics}
%% }

%% Uncomment below to include a teaser figure.
\teaser{
  \centering
  %% \includegraphics[width=.9\linewidth]{figs/teaser.jpg}
  % \vspace{-.5em}
  \caption{We'll tease you.
  }
% \vspace{-.5em}
}

%% Uncomment below to disable the manuscript note
%\renewcommand{\manuscriptnotetxt}{}

%% Copyright space is enabled by default as required by guidelines.
%% It is disabled by the 'review' option or via the following command:
% \nocopyrightspace

%
% Useful Macros 
%

\newcommand{\eg}{e.g.\xspace} % e.g. and i.e. ARE NOT ITALIC!!
\newcommand{\ie}{i.e.\xspace}
\newcommand{\todo}[1]{\textcolor{red}{#1}}

\newcommand{\jim}[1]{{\color{green} Jim: [{#1}]}}
\newcommand{\carlos}[1]{{\color{blue} Carlos: [{#1}]}}
\newcommand{\lauro}[1]{{\color{red} Lauro: [{#1}]}}


%%%%%%%%%%%%%%%%%%%%%%%%%%%%%%%%%%%%%%%%%%%%%%%%%%%%%%%
%%%%%%%%%%%%%%%%%%%%%% ALGORITHM %%%%%%%%%%%%%%%%%%%%%%
%%%%%%%%%%%%%%%%%%%%%%%%%%%%%%%%%%%%%%%%%%%%%%%%%%%%%%%

\renewcommand{\algorithmicthen}{}

% aux. commands

%%%%%%%%%%%%%%%%%%%%%%%%%%%%%%%%%%%%%%%%%%%%%%%%%%%%%%%%%%%%%%%%
%%%%%%%%%%%%%%%%%%%%%% START OF THE PAPER %%%%%%%%%%%%%%%%%%%%%%
%%%%%%%%%%%%%%%%%%%%%%%%%%%%%%%%%%%%%%%%%%%%%%%%%%%%%%%%%%%%%%%%%

\begin{document}

%% The ``\maketitle'' command must be the first command after the
%% ``\begin{document}'' command. It prepares and prints the title block.

%% the only exception to this rule is the \firstsection command

\firstsection{Introduction}

"Coordination is done by meetings"

"how do you trace an automated alert back to an EDA environment ?"


\maketitle

\section{Related Work}

Kandel et al. argue in their review of data wrangling work that data
cleaning, wrangling and transformation is a major part of exploratory
analysis and visualization~\cite{Kandel:2011:RDI}. Although RCloud
does not by itself include modules specific to data cleaning and
wrangling, we note from internal experience that these cleaning
scripts themselves tend to change over time. RCloud addresses this
issue by providing easy publishing of data-cleaning notebooks as web
services, which reduces the total data-cleaning effort across an
organization.

Kandel et al.'s interview study points out the typical ``explore'',
``model'', ``report'' cycle in enterprise data
analysis~\cite{Kandel:2012:EDA}. There are many discontinuities in
this cycle that cost time and effort to overcome.  RCloud seeks to
reduce this impedance mismatch. They also point out that larger teams
are becoming more common in data analysis, that supporting
collaboration is a difficult and important problem, and that sharing
and versioning of data sources and artifacts is hindered by current
technology in practice. ``We found that analysts typically did not
share scripts with each other. Scripts that were shared were
disseminated similarly to intermediate data: either through shared
drives or email. Analysts rarely stored their analytic code in source
control.'' Their work points to the opportunity for better technology
to support collaboration and sharing by data analysis teams.

Heer and Agarwala identify many design considerations for
collaborative visual analytics~\cite{Heer:2008:DCF}. \emph{Starring},
the system for signaling interest in notebooks, described in
Section~\ref{sec:starring}, addresses social-psychological incentives,
recommendation, and voting and ranking. RCloud's integrated deployment
mechanism, described in Section~\ref{sec:deployment}, addresses cost of
integration, content export, presentation and view sharing. Notebooks,
and the integrated version control system for them, described in
Section~\ref{sec:notebooks}, address modularity and granularity, and
artifact histories.

Manyeyes \cite{Viegas:2007:MAS} was a landmark system supporting the
integration of social media with visualization and data publishing.

\cite{Perer:2008:ISA}

Our work has been inspired by and has benefited from other recent projects
to improve data analysis environments and processes, including RStudio,
R packages such as Markdown, knitr and Shiny, and iPython notebooks,
to name a few. RStudio aims at providing an integrated development environment
for R programming, with support for publishing code in packages. Markdown,
knitR and Shiny provide R with sophisticated reporting capabilities, including
interactive web interfaces. IPython \cite{Perez:2007:IAS}
shares many of our goals...

\section{System Architecture}

\section{System Design}

\subsection{Notebooks\label{sec:notebooks}}

\subsection{Reputation and Interest: starring\label{sec:starring}}

\subsection{Deployment of notebooks\label{sec:deployment}}

Notebooks as versioned subroutines, web services.

Notebooks by default are visible by the entire organization: there
exists a URL for every notebook in RCloud. This is deliberate. As
pointed out by Wattenberg and Kriss~\cite{Wattenberg:2011:DFS}, broad
access to analysis outputs (in their case, in the form of NameVoyager)
increases long-term engagement partly by the crossreferences in the
web. Our main RCloud deployment is only visible inside an intranet,
but we have nevertheless found anecdotal support for this theory by
noticing links to RCloud notebooks in internal discussion fora and
mailing lists.

\section{Case Studies}

\section{Lessons Learned}

\section{Discussion and Limitations}

% limitations
Previous studies have pointed out difficulties in achieving production-level
flexibility, scalability and maintainability with experimental code created
by data analysts. These properties cannot be ensured by any programming environment,
but suitable tools can help well-intentioned analysts and programmers to create
robust applications with less effort.

Asynchronous collaborative visual analytics
(ACVA)~\cite{Chen:2011:SEC}. This paper addresses visualization
\emph{of} the ACVA process. It will be important when we talk about
recommendation systems, and navigation of the set of notebooks, etc.

\bibliographystyle{abbrv}

\bibliography{paper}
\end{document}
