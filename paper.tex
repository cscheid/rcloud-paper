%\documentclass[journal]{vgtc}                % final (journal style)
\documentclass[review,journal]{vgtc}         % review (journal style)
%\documentclass[widereview]{vgtc}             % wide-spaced review
%\documentclass[preprint,journal]{vgtc}       % preprint (journal style)
%\documentclass[electronic,journal]{vgtc}     % electronic version, journal

%% Uncomment one of the lines above depending on where your paper is
%% in the conference process. ``review'' and ``widereview'' are for review
%% submission, ``preprint'' is for pre-publication, and the final version
%% doesn't use a specific qualifier. Further, ``electronic'' includes
%% hyperreferences for more convenient online viewing.

%% Please use one of the ``review'' options in combination with the
%% assigned online id (see below) ONLY if your paper uses a double blind
%% review process. Some conferences, like IEEE Vis and InfoVis, have NOT
%% in the past.

%% Please note that the use of figures other than the optional teaser is not permitted on the first page
%% of the journal version.  Figures should begin on the second page and be
%% in CMYK or Grey scale format, otherwise, colour shifting may occur
%% during the printing process.  Papers submitted with figures other than the optional teaser on the
%% first page will be refused.

%% These three lines bring in essential packages: ``mathptmx'' for Type 1
%% typefaces, ``graphicx'' for inclusion of EPS figures. and ``times''
%% for proper handling of the times font family.

\usepackage{mathptmx}
\usepackage{graphicx}
\usepackage{times}
\usepackage{xspace}
\usepackage{amssymb}

% \usepackage[onecolumn]{multicol}

%% algorithm packages (llins)
\usepackage{algorithm}
\usepackage{microtype}
\usepackage{algpseudocode}

%\usepackage{dblfloatfix}
%\usepackage{fixltx2e}
\usepackage{url}
\usepackage{scrextend}   % for addmargin in introduction.tex
\usepackage[usenames,dvipsnames]{xcolor}


%% We encourage the use of mathptmx for consistent usage of times font
%% throughout the proceedings. However, if you encounter conflicts
%% with other math-related packages, you may want to disable it.

%% This turns references into clickable hyperlinks.
\usepackage[bookmarks,backref=true,linkcolor=black]{hyperref} %,colorlinks
\hypersetup{
  pdfauthor = {},
  pdftitle = {},
  pdfsubject = {},
  pdfkeywords = {},
  colorlinks=true,
  linkcolor= black,
  citecolor= black,
  pageanchor=true,
  urlcolor = black,
  plainpages = false,
  linktocpage
}

%% If you are submitting a paper to a conference for review with a double
%% blind reviewing process, please replace the value ``0'' below with your
%% OnlineID. Otherwise, you may safely leave it at ``0''.
\onlineid{341}

%% declare the category of your paper, only shown in review mode
\vgtccategory{Research}

%% allow for this line if you want the electronic option to work properly
\vgtcinsertpkg

%% In preprint mode you may define your own headline.
%\preprinttext{To appear in an IEEE VGTC sponsored conference.}

%% Paper title.

\title{Collaborative Visual Analysis on the Web with RCloud}
% \title{Integrating Social Visual Analysis and Deployment with RCloud}
% \title{Social Visual Analytics: DevOps for Data Scientists}

%% This is how authors are specified in the journal style

%% indicate IEEE Member or Student Member in form indicated below
\author{Stephen North and Carlos Scheidegger and Simon Urbanek and
  Gordon Woodhull}

\authorfooter{
%% insert punctuation at end of each item
\item 
Stephen North directs Infovisible LLC, Oldwick, USA, and graphviz.org.
\\Email: s.c.n@ieee.org
\item
Carlos Scheidegger is with the University of Arizona, USA.
\\Email: cscheid@email.arizona.edu
\item
Simon Urbanek is with AT\&T Labs in Bedminster, USA.
\\Email: urbanek@research.att.com
\item
Gordon Woodhull is with AT\&T Labs in Bedminster, USA.
\\Email: gordon@research.att.com
}

%other entries to be set up for journal
% \shortauthortitle{Biv \MakeLowercase{\textit{et al.}}: Global Illumination for Fun and Profit}
%\shortauthortitle{Firstauthor \MakeLowercase{\textit{et al.}}: Paper Title}

%% Abstract section.
\abstract{ Consider the emerging role of data science teams embedded in
  larger organizations. Individual analysts work on loosely related problems,
  and must share their findings with each other and the organization at large,
  moving results from exploratory data analyses (EDA)
  into automated visualizations, diagnostics and reports deployed for wider consumption.
  There are two problems with the current
  practice. First, there are gaps in this workflow: EDA is performed
  with one set of tools, and automated reports and deployments with another.
  Second, these environments often assume a single-developer perspective,
  while data scientist teams could get much benefit from easier sharing
  of scripts and data feeds, experiments, annotations, and automated
  recommendations, which are well beyond what traditional version control
  systems provide. We contribute and justify the following three
  requirements for systems built to support current data science teams
  and users: \emph{technology transfer}, \emph{coexistence}, and
  \emph{discoverability}. In addition, we contribute the design and
  implementation of RCloud, a system that supports the requirements of
  collaborative data analysis, visualization and web deployment.
  The biggest deployment of RCloud has been in active use for more than two years,
  and has about fifty active users. We report on interviews with some of these users, and
  discuss the design decisions, tradeoffs and limitations, comparing RCloud
  to other current proposals.
} % end of abstract

%% Keywords that describe your work. Will show as 'Index Terms' in journal
%% please capitalize first letter and insert punctuation after last keyword
\keywords{visual analytics process, provenance, collaboration, visualization, computer-supported cooperative work}

%% \CCScatlist{ % not used in journal version
%%  \CCScat{K.6.1}{Management of Computing and Information Systems}%
%% {Project and People Management}{Life Cycle};
%%  \CCScat{K.7.m}{The Computing Profession}{Miscellaneous}{Ethics}
%% }

%% Uncomment below to include a teaser figure.
\teaser{
  \centering
  \includegraphics[width=.95\linewidth]{fig/teaser/teaser.pdf}
  % \vspace{-.5em}
  \caption{
    An overview of features in RCloud. RCloud supports an environment in which a large number of loosely-related problems in data and visual analytics are solved by a small number of \emph{hackers} and \emph{scripters}. In such a shared environment, \emph{discoverability} is a concern. RCloud supports search, annotation, recommendation, and commenting for all notebooks, and provides an overview where users can \emph{browse} popular and recent analyses. When problems and data sources change frequently, deployment can become very costly; RCloud supports \emph{transparent and automatic} deployment of analyses as web pages and web services, allowing a seamless transition from data exploration work to production web services.
  }
% \vspace{-.5em}
}

%% Uncomment below to disable the manuscript note
%\renewcommand{\manuscriptnotetxt}{}

%% Copyright space is enabled by default as required by guidelines.
%% It is disabled by the 'review' option or via the following command:
% \nocopyrightspace

%
% Useful Macros
%

\newcommand{\eg}{e.g.\xspace} % e.g. and i.e. ARE NOT ITALIC!!
\newcommand{\ie}{i.e.\xspace}
\newcommand{\todo}[1]{\textcolor{red}{#1}}

\newcommand{\jim}[1]{{\color{green} Jim: [{#1}]}}
\newcommand{\carlos}[1]{{\color{blue} Carlos: [{#1}]}}
\newcommand{\lauro}[1]{{\color{red} Lauro: [{#1}]}}


%%%%%%%%%%%%%%%%%%%%%%%%%%%%%%%%%%%%%%%%%%%%%%%%%%%%%%%
%%%%%%%%%%%%%%%%%%%%%% ALGORITHM %%%%%%%%%%%%%%%%%%%%%%
%%%%%%%%%%%%%%%%%%%%%%%%%%%%%%%%%%%%%%%%%%%%%%%%%%%%%%%

\renewcommand{\algorithmicthen}{}

% aux. commands

%%%%%%%%%%%%%%%%%%%%%%%%%%%%%%%%%%%%%%%%%%%%%%%%%%%%%%%%%%%%%%%%
%%%%%%%%%%%%%%%%%%%%%% START OF THE PAPER %%%%%%%%%%%%%%%%%%%%%%
%%%%%%%%%%%%%%%%%%%%%%%%%%%%%%%%%%%%%%%%%%%%%%%%%%%%%%%%%%%%%%%%%

\begin{document}

%% The ``\maketitle'' command must be the first command after the
%% ``\begin{document}'' command. It prepares and prints the title block.

%% the only exception to this rule is the \firstsection command


\firstsection{Introduction}

\maketitle

\todo{Points we want to make."Coordination is done by meetings" "how do you trace an automated alert back to an EDA environment ?"}

% What's the area
Consider the emerging role of a data science team within an
organization today \cite{Keim:2008:VAS}.

% How do people do it today?
Individual data scientists and statisticians usually work on loosely
related problems, and must devise effective ways to share their findings
and move results from exploratory data analysis to automated diagnostics
and reports that are deployed for wider consumption.
They generally rely on a patchwork of resources shared informally,
or at best, through systems that attack individual tasks or functions
in the data analysis workcycle. % might cite information visualization reference model or pirolli/card

% Why does this not work?
There are two problems with the current
practice. First, there are gaps in the workflow:
Exploratory Data Analysis (EDA) is done with one set of tools,
and automated reports and deployments with another. This makes
transferring results from EDA to production slow, expensive, and error-prone.
Second, EDA environments often assume a single-developer perspective,
while data scientist teams could greatly benefit from convenient sharing of
scripts, data sets, data feeds, experiments, annotations, and automated
analyses and recommendations. These features are well beyond what existing
repositories and version control systems provide.

% What are we going to do about this?
\todo{Sell idea, not system} In this paper, we investigate the hypothesis
that the practice of visual analytics can advance by adopting techniques
from information retrieval and collaborative work environments. To investigate
these ideas, we designed and deployed a prototype environment called RCloud,
that supports collaborative data analysis, visualization and web deployment.
We will report on the design decisions, tradeoffs and limitations, and compare
the concepts in RCloud to similar proposals.

\todo{Define the requirements for the system}

We begin by noting the main requirements for a data analysis environment for
collaboration and transfer of results into production.

\begin{itemize}
\item Technology transfer. Experiments in notebooks developed by analysts need to become accessible in the production environment, making it easy to push new features, prototypes, etc.
\item Dashboarding. To support the above, there should be a mode in which non-experts
can operate the experiments without coding.
\item Version control as a primary concept in the repository. There needs to be a way to control the visibility of prototypes as they are being developed and tested.
\item Modern interactive technologies. People want to interact with data to explore it. Scalable technologies like crossfilter exist. Want to support those in the context of an R environment. Need to support full 2-way communication between computing in the cloud and interaction on the web.
\item Other requirements e.g. Kandel's user study. Not our contribution but things we must address, mentioned in related work section. Heer/Agrawala design considerations for collaborative visual analytics.
\end{itemize}

% What is the scope of our solution?
Our investigation focuses on the R language \cite{RCoreTeam:2013:R}
for statistical computing and graphics. R is one of the most popular
systems for ...

% We want our solution to play well with the rest of the ecosystem
R has developed a complex ecosystem of compatible packages, software
tools and other resources. For example there are packages that make
it convenient to execute map-reduce programs, generate web interfaces,


\section{Related Work}

Kandel et al. argue in their review of data wrangling work that data
cleaning, wrangling and transformation is a major part of exploratory
analysis and visualization~\cite{Kandel:2011:RDI}. Although RCloud
does not by itself include modules specific to data cleaning and
wrangling, we note from internal experience that these cleaning
scripts themselves tend to change over time. RCloud addresses this
issue by providing easy publishing of data-cleaning notebooks as web
services, which reduces the total data-cleaning effort across an
organization.

Kandel et al.'s interview study points out the typical ``explore'',
``model'', ``report'' cycle in enterprise data
analysis~\cite{Kandel:2012:EDA}. There are many discontinuities in
this cycle that cost time and effort to overcome. RCloud seeks to
reduce this impedance mismatch. They also point out that larger teams
are becoming more common in data analysis, that supporting
collaboration is a difficult and important problem, and that sharing
and versioning of data sources and artifacts is hindered by current
technology in practice. ``We found that analysts typically did not
share scripts with each other. Scripts that were shared were
disseminated similarly to intermediate data: either through shared
drives or email. Analysts rarely stored their analytic code in source
control.'' Their work points to the opportunity for better technology
to support collaboration and sharing by data analysis teams.

Heer and Agarwala identify many design considerations for
collaborative visual analytics~\cite{Heer:2008:DCF}. \emph{Starring},
the means for signaling interest in notebooks, described in
Section~\ref{sec:starring}, addresses social-psychological incentives,
recommendation, and voting and ranking. RCloud's integrated deployment
mechanism, described in Section~\ref{sec:deployment}, addresses cost of
integration, content export, presentation and view sharing. Notebooks,
and the integrated version control system for them, described in
Section~\ref{sec:notebooks}, address modularity and granularity, and
artifact histories.

Manyeyes \cite{Viegas:2007:MAS} was a landmark system for the integration
of social media with visualization and data publishing. We build on this
work by defining a rich interface for collaboration about code and for
operating on metadata.

The need for integrating statistics and visualization has been
highlighted in previous studies and is widely understood by
various technical communities \cite{Perer:2008:ISA}

There has been noteworthy work on specific techniques such as
social bookmarking \cite{Millen:2006:DSB} \cite{Heer:2007:VAV}
and crowdsourcing \cite{Fast:2014:ECS} to support collaborative
or social development or analysis processes.
Similarly, there are computational methods to support high
performance execution in incremental code development
environments \cite{Guo:2010:TPI}.
Our goal is to define an environment in which many such
techniques may be integrated and made available to a broad community.

VisMashup~\cite{Santos:2009:VST} and Crowdlabs~\cite{Mates:2011:CSA}
are tools built on top of the VisTrails workflow management
system~\cite{Callahan:2006:VVM}. VisMashup defines a schema and
semantics for automatically deriving user interfaces from workflows,
while Crowdlabs exposes these capabilities on a website feature
workflow upload and remote execution. In our view, the impedance
mismatch between a dataflow pipeline specification and the power of a
general-purpose language is too great for the type of general
exploratory work in data science teams. As a result, RCloud tries to
provide a closer data-analysis to analysts accustomed to creating and
executing R and Python code while keeping many of the attractive
properties like automatic versioning and management.

Our work has been inspired by and benefited from other proposals
to improve data analysis environments and processes,
including RStudio \cite{RStudio:2013:SWA},
R packages such as Markdown \cite{Allaire:2014:MMR},
knitr \cite{Xie:2013:DDW}
and Shiny \cite{RStudio:2013:SWA},
and iPython notebooks \cite{Perez:2007:IAS}
to name a few. RStudio aims at providing an integrated development environment
for R programming, with support for publishing code in packages. Markdown,
knitR and Shiny provide R with sophisticated reporting capabilities, including
interactive web interfaces. IPython \cite{Perez:2007:IAS}
shares many of our goals, such as providing a comprehensive environment
for analysis and programming, with shareable documents on the web.

One overall goal is to reduce the gap between implementers and deployers
of technology. Don't want a team of 20 IT people to support 5 data scientists.

% devops for data science.

\section{The System\label{sec:system}}

\subsection{High-level Architecture\label{sec:highlevelarchitecture}}

% Stephen: Let's work from front to back, from human interface to the
% services and frameworks that support it.

The internal computing infrastructure of organizations has changed
radically in the last fifteen years. The shift from large
servers toward scalable, lower-cost, distributed systems (``the cloud'')
led to a software ecosystem of processes distributed over a
network, usually communicating via HTTP. HTTP is dominant because
web browsers and servers are ubiquitous and available in almost
all hardware devices, from tiny sensors, to handheld devices and
laptops, to rack-mounted servers.

As a result, HTTP is the lingua franca of interprocess communication
(IPC). One of the design goals for RCloud was to provide a productive
environment for creators of data-analysis scripts, that also behaves
as a first-class citizen in the pre-existing ecosystem of computer
services and networks in an organization.

%% Design for cloud-friendly? This goes back to the point in the
%% introduction about playing nice with the rest of the ecosystem.

\subsection{Human Interface\label{sec:humaninterface}}

%\stephen{What??}
%We designed RCloud around the front-facing API, which provides roughly
%one entry point to correspond with each requirement.

Figure~\ref{fig:something} shows the RCloud developer interface.
The center panel is a notebook for code and visualizations.
Code is edited in this panel, and visualizations are rendered.
An inventory of supplementary ``asset'' files, that are part of
the notebook, though not in its executable flow, are edited on the right.
Controls at the top of the screen allow the user to run, view, and
share the notebook. The result of code execution is the final cell of
the notebook.  On the left is a browsing area for searching and viewing
other users' notebooks, and a help system.

The call mechanism is the URL itself; arguments are read from the URL
and the final cell is rendered as the result. 

\subsection{Notebooks\label{sec:notebooks}}

The main unit of computation in RCloud is a \emph{notebook}.
A notebook holds a sequence of \emph{cells}, each of which contains a
snippet of code or hypertext in Markdown. This is not a novel idea;
executable documents structured this way are a feature of many
other systems, including Mathematica, IPython and Sage.
\stephen{Code is executed when...?}

One of the main contributions of RCloud is the idea that notebooks
are ``always deployed''. In other words, the most recent version of
a notebook is immediately available.\stephen{What if it is broken?}
This makes it convenient to share and modify experiments and compose
results without binding to a specific version of a notebook.
On the other hand, there are situations where it might be important
to name specific versions. We do not expect designers to decide which
versions need to be preserved, but we embrace \emph{transparent} versioning.
This is similar to models like Jankun-Kelly et al.'s p-set calculus \cite{Jankun-Kelly:2007:MFV}
and VisTrails's version tree \cite{Callahan:2006:VVM}, where every change in the state of the system is tracked.

To implement this, we built RCloud on top of Github's \emph{gists}~\cite{Github:2014:GG}.
Github offers a HTTP interface for creating simplified git repositories, the main limitation
being a restriction to text-only files in a single directory. The GitHub web-service
API provides most of the semantics we need for the versioning portion of the storage back end:
access to previous versions, comments, starring, and forking.

Using GitHub for storage and versioning also exposes other capabilities
that can be invoked with JSON and HTTP.
Particularly, we can provide full text search using Apache SOLR.
SOLR is scalable, can index in near-realtime, supports multiple character sets,
indexes several common types of documents, and has schemas and faceted search.
%
Integrating existing services and software components, especially open source,
instead of implementing custom code is a trend in the software industry, and
is very positive for systems research and prototyping in visual analytics.
By adopting existing technology, small teams with limited resoures can explore
novel ideas toward the improvement of large, complex software ecosystems.
In our case, even though this approach involved learning standards not directly
related to visual analytics, our early decision to use the GitHub gist API
turned out to be successful. Future projects could gain many of the same benefits
by adopting this strategy. In fact, it will be essential for technology adoption
and transfer, because it is almost impossible for any one tool or platform to
``own'' or support the entire visual analytics process.
\stephen{Are we preaching too much?}

\subsection{Reputation and Interest: starring\label{sec:starring}}

Information retrieval based on collecting usage and recommendations
is a cornerstone of modern web services. We would like to help data
scientists to find workbooks (and therefore items in their contents
such as code, data, and colleagues with specific expertise) with
the benefit of such information.

In RCloud, reputation and interest are a relationship between
\emph{notebooks} and \emph{users}, rather than a relationship between
user pairs. We chose this approach because we expect initial 
RCloud deployments to have relatively few users, but some users to
create many notebooks. Under that assumption, assigning interest
to users would not provide sufficiently ``high-resolution'' data.

We incorporate both explicit and implicit indications of interest
in notebooks. Explicit interest is indicated by ``starring,'' or
clicking on a button that marks a notebook as interesting.
This makes explicit indication of interest a nearly trivial operation,
always readily available, to encourage its use.

Implicit signaling of interest is supported by keeping click-through counts
\cite{Joachims:2005:AIC} and execution counts. (In addition to these
standard techniques of collecting feedback from web search, we anticipate
applying static and dynamic code analysis to infer fine-grained
information about relationships, for example, which packages and data
sets often appear together.)

\subsection{Executing R in a web browser\label{sec:Rinbrowser}}

One our main goals is to provide ubiquitous access to the statistical
tools of the R programming language in the wider environment in which data
science teams participate.
%
Given the recent developments of interactive visualization and visual
analytics in HTML5 and the web, we decided to create an
\emph{interconnect} between the two environments.
%
Every RCloud session spawns a new R execution process in a remote
server.

%% \subsubsection{Object capabilities: R to Javascript RPC and access
%%   control}

HTTP and web services are convenient but much of the protocol is
stateless. (For example, GET requests are required to not change the
remote state, and results can be cached in transparent proxies along
the network). In the case of close communication between a running
R process and a web browser, we need something that is more flexible
than distinct URLs for every R process, and which embraces
statefulness of both calculations and visualization parameters.

\carlos{In-depth explanation of R<->web interconnect, and why it's
  necessary or important. Main point: make programming in the R side
  as close as possible to what R programming feels like, and make
  programming in the Javascript side as close as possible to what
  Javascript programming feels like. We do this because this part of
  the system is targeted at programmers; Shiny targets web programming
  at non-web-programmers. We don't attempt such a thing and land at a
  different point in the spectrum.}

\carlos{compare against https://www.opencpu.org, compare against
  IPython notebooks. Two-way communication? Actually describe what it
  is. IPython manipulate notebooks? Understand and contrast}

\subsection{Interactive notebooks\label{sec:interactivenotebooks}}

%% How do we do things that are not trivial to do with IPython (for
%% example)

%% What is relationship to a standard CMS with R/Python

%% dcplot. two-way communication between between backend session and
%% frontend session.

A key engineering decision in RCloud was to rely on full two-way
communication between the web client human interface, and data and
computing resources in the cloud. It is easier to engineer a design
in which the human interface is pushed one-way from the server
and thereafter all interaction takes place within the client.
Two-way communication is necessary, though, where the size of
the data or computational performance makes it impractical to
move all computation to the client.

In addition, R is well suited as a language for analysis, and
JavaScript for interactive visualization. To draw on
the strengths of both languages and both environments, the connection
between the languages is not just procedural: it makes \emph{closures}
and \emph{first-class functions} available across the network.
This provides considerable flexibility, so that for example, a chart built
with dc.js or leaflet.js can call back to analysis functions in R
without having to formalize the protocol between the processes.

\subsection{Deployment of notebooks\label{sec:deployment}}

Every notebook in RCloud is named by a URL, and notebooks by
default are visible by the entire organization. This is deliberate.
As pointed out by Wattenberg and Kriss~\cite{Wattenberg:2011:DFS},
broad access to analysis outputs (in their case, for NameVoyager)
increases long-term engagement in part through cross-references on
the web. Although our prototype RCloud deployment is only visible
inside a corporate intranet, we nevertheless found anecdotal support
for this notion by discovering links to RCloud notebooks in internal
discussion fora and mailing lists.

Because notebooks are continually published, any work is immediately
available both as a subroutine and a visual component. Close colleagues
can start on the next stage of analysis, or delve into the data,
even while the original author is polishing an algorithm or its
presentation. \stephen{We glossed over the things that can go wrong.
We should at least discuss that briefly.}
The code is the page, and can either be shaped into
a function of inputs and outputs, or the linear cells of the notebook
can be reworked into a full-fledged HTML layout.

\section{Case Studies}

In this section we present a pair of example applications of the
capabilities provided by RCloud. The examples themselves are synthetic
(unfortunately, we are unable to show production notebooks because of
their business value), but the examples are representative of the
type of progression we want the ecosystem to support.

\subsection{Stock Price Analysis\label{sec:stockvis}}

\begin{figure}
\includegraphics[width=\linewidth]{fig/casestudy1/casestudy1.pdf}
\caption{\label{sec:stockvis}Iterations of a stock ticker
  visualization, based on an example suggested by Hadley Wickham.  A
  simple, static visualization of the closing price of a single ticker
  is progressively improved into a configurable visualization suitable
  for dashboarding, then into an interactive visualization of the
  volatility and volume of two separate stocks, then finally into API
  calls for data access use in other RCloud notebooks. All notebooks
  in this can be used directly as web pages. When a notebook
  corresponding to a function call is used as a web page, its included
  documentation is displayed.}
\end{figure}

Our first example is a sequence of visualizations of the performance
of stock prices over a multi-year period. The first visualization uses
ggplot2 and is due to Hadley Wickham. It uses data provided by Yahoo!
Finance via their web service, and shows the stock price for a single
symbol.

The webpage that produces that visualization always includes a link
back to the generating source code, from which a user can always fork
the notebook and, in this case, add a configurable ticker based on the
URL of the notebook. Notice how the notebook changes very little.

From there, the notebook author (or a user) can decide to provide easy
access to the data without producing a visualization. This is achieved
by simply creating a notebook that defines a function. This notebook
then becomes a \emph{subroutine} for other notebooks, and is
version-controlled in the same way.

Interactive version.

%% This notebook can then be called in an \emph{interactive}
%% visualization, which uses the Javascript 

%% Easy to convert it into configurable entry point for dashboarding.

%% Now consider an analyst that wants to understand the price dynamics of
%% different stock attributes, different points in time, etc. For that,
%% interactive visualization is very attractive. describe dcplot and
%% notebook which uses it.

%% Then describe abstraction of stock-data fetching, to talk about
%% calling notebooks from other notebooks.

%% then call.r?

\subsection{text analysis\ref{sec:textvis}}

Same.

% IMPORTANT: what is unique about RCloud here
% From prototyping to dashboard
% Getting other information from the web
%

\section{Interview Study\label{sec:interviews}}

To evaluate the effectiveness of RCloud, we interviewed 13 current and past users of RCloud. Of these, 9 are adata analysts, and 4 build tools for data analysts and business needs.

\gordon{Add user statistics here: so many users use the Research instance a lot, so many use it an intermediate amount, so many just tried it out. Simon to work up numbers on this. Also will get stats on number of notebooks, forks, stars, from Prateek's notebook.}


\subsection{Sharing of results}
Rick: ``I like being able to look at other people's notebooks.''

Clint: ``I can share it with people and copy it pretty readily. So if my supervisor wants to see what I've done or QA it, I can just send her a link.'' ``I think the best part about it is how easily you can share code... you can find a working example, rather than wearing out Google and finding questionable examples that may or may not work.''

Corey: ``being able to share notebooks or codes, or collaborate on something, easily being able to share that information. That's an obvious advantage and nice to have.''

Raif: ``I've been sharing my notebooks in order for other people to see what I've done. It's very convenient for that purpose, I would say'' ``we haven't been editing them together, but we've been just sharing, and then I've been looking at these things that other people do. So there's no collaborative aspect of it, like developing the code, but there's like sharing aspects for other people to see and for me to see other people's notebooks'' ``the fact that we have all the notebooks there, searchable, it just saves me from replicating what other people have done'' ``there already is some person who has done something similar, and then you're able to just edit that, and that's saved a lot of work time for me''

Shelly: Is unable to use RCloud for organizational reasons but likes the ``concept of being able to create notebooks and share them'' ``wiki is not the best way for communication results - it's kind of like writing a blog post with very limited functionality. can't input equations easily.'' ``have to save every picture and post it as an image'' ``I can't share all of the code because it would just get crowded and wouldn't look right on a wiki''

Bart: Also unable to use for org/tech reasons. ``So if I could make a folder on RCloud and have Python notebooks and also Pig notebooks there, and execute them from RCloud, that would be much better than my current thing, because that would free me from manual documentation and version control and also telling people where my code was. It would be just, hey, go look on RCloud, here's the stuff.''

Horace: ``It's not for myself, but for other people who use RCloud vs RStudio, they will always think that they want to prototype on RStudio. Maybe because it's more familiar for them. So, they work on RStudio, on a small data set, and they fix up everything, make sure everything is right, and then they place it on RCloud for sharing.''

Taylor: ``I use it more for sharing code with other people, and for doing tutorials for iotools or hmr or explaining'' ``a lot of times I'll share the code, because I'm doing so much package development. I want people to see how the package works, so I clearly want them to see the code… I kind of write it just like I would write GitHub Markdown, where you have little code snippets and text, but RCloud lets me actually run the snippets” with results''


\subsection{Forking}
Rick: ``I fork my own notebooks because I'm going off and doing some other analogous project, so I've sort of got interesting content that I've already done in a previous analysis that I want to start from and then tweak to match a new set of data.''

Clint: ``I use [forking] on my own work quite a bit, and I use it on Research a ton. Honestly, it's one of the handiest functions in it, because instead of having to find it, copy, paste it, you just hit Fork, you rename it, and it's done. It's pretty amazing.''

Raif: ``I've been forking other people's notebooks myself, because essentially they have done some work, but then I want to run it on a different part of data, or I want to change some parts, I don't want to see this column, I want a different column, things like that''

Shelly: ``gives the audience a way to try their own analysis, like maybe Shelly should have used this parameter and they can just change the parameters''

Horace: on working with other group using his notebooks, ``I basically point them to a few different parameters that they need to fix.'' ``I'll teach people to intercept the result in the middle and see what you should see, and what's not being returned, that sort of thing'' ``If the computation went wrong in the middle, and something blew up, or you're just returning NULL or empty data, where should we check those things?'' ``We should, but a lot of the time we just insert print statements here and there and check values. Having debugging code inside a functioning program is actually helpful for others to understand what you're doing?''


\subsection{Automatic source control}
Rick: ``when I've finished something, there's a nice clean record of it.''

Clint: ``instead of looking back and saying I've got a billion files here in this subdirectory and I hope I've got them backed up, if they're on RCloud I know they are'' ``I don't have to worry about it if I drop my computer in the bathtub, everythings gone, because I'm not great about uploading my own for versioning, right?''

Bart: ``I like the fact that it has a built-in editor, so if you need to, like, fixing a typo in a link, or an extra line break or any of this other nonsense, you can just switch to the edit view, pull up your asset, type something, it's automatically saved, committed, everything. You don't have to go back to your source code, change it, commit it to the repo, you pull the repo to your distribution version, none of that's necessary. A lot of web development is fixing stupid little bugs, and you can do that kind of instantly without a lot of overhead, which is nice.''

\subsection{Web integration}
Horace ``that also means the other guys who want to do analytics on the data, they can first pull the data, do the analytics on it, and then feed the viewer the data. So if anyone from the stats group wants to insert some code in between the data pulling procedure - now it's just database to GUI, and you can always insert something in between. We were trying to make such an example so that they don't have to rewrite the GUI part and they know how to get the data source, once you get the data into RCloud, then you have a dataframe to work with, and then they know how to produce another dataframe. We give them the specification of the dataframes, we give them the data input, they do whatever they want, they can reuse the data connection part and the GUI part. We can keep changing these components and that's a very nice thing.''

Jyothi: ``Having an open session with R certainly helps too, because both of my applications need to be doing something with the data, so having an open session where I can run R command or R functions without having to invoke an API or send out a request and then wait for the request to come back to do the results.  So that kind of seamless integration is extremely helpful in writing the application.''

Jyothi: ``another thing I really like about RCloud is your publish feature, so that non-RCloud users, the anonymous login, non-RCloud users can access the notebooks without having to login to RCloud''

accidental feature: ``some of my jobs are huge and I like to just call them and the fact that the user can just shut down his browser or shut down his computer and the job still runs in the background is actually a very neat thing. So what, um, the way I have coded is, whenever I have a huge job to run, I just display to the user, you go do whatever you want, we'll send you an email when the job is done, um that's another feature of RCloud that's very helpful, that the job will continue to run in the background''


\subsection{Proliferation of notebook cells}
Rick: ``Making new cells and taking some stuff in it, then having it do something, then having to make a new cell, and I don't know, it just gets in my way, so if I'm doing really straightforward simple stuff, I'll just get into R and do it myself.'' ``I'm always scrolling up and down the screen'' ``every time I type more stuff, the notebook gets longer and longer and it's harder to deal with'' ``I've got to do a lot of scrolling around to get to where I'm going, whereas if I'm doing it in vi, I can search for things in my typing, and get to places I want to be without having to scroll through it and try to read it'' ``times when I just want to type in a couple of quick and get some results that are going to tell me what to do next, and they're not necessarily archival in any sense'' ``being literate about anything typically takes a lot of rewriting and going back over things... I just want to explore a few things and then I'll know what I want to write.'' ``I'll often type in expressions that allow me to check that I'm the right track, so I know that if the computation worked out, then the sum of the residuals is gonna be zero. So I'll type in sum(resid) and if it prints out zero, then I'm all happy, but I don't want that in my notebook.'' ``it tempts me to type a big long thing and then run the whole thing, as opposed to, type a few little pieces that I'm thinking of on the inner parts and then put them together to make the big thing''

Ivan has the complaint that too many cells are created to use the notebook as a console.  This has a lot to do with the extra real estate that gets taken up by the cells controls, blank space, etc of cells versus a straight command line.

KC and Ivan both complained that when there are long results or plots, it causes the code to scroll off the screen.

Kenny: ``It saves everything I do like everything is gold, but most of it is junk not meant to be saved'' ``the console is purely disposable''

Raif: ``cells are really useful'' but ``whatever you type on the command line, it becomes a part of your notebook, and I find it a little bit annoying, because I really type a lot of things to check my data, and I don't want that to become a part of the notebook'' ``Like you're kind of checking out what your data is, or you make a plot of the data. Things that should not really become part of a notebook, but things that help you understand your data better.'' ``for example imagine I run a cell whose output is huge, right, and now that output would block the whole window, and in order to find the next cell, I have to scroll down, and find where that cell starts, right? So, I'm losing the continuity of my code to the output''

Horace: ``Well, the nice thing is if you have the whole notebook, you can run it step by step, and try to mess around in between, interactively.''

Rick on auditing (possible solution): ``you didn't see your mistakes, they were in there but you never really saw them because a mistake never led to an answer at the end'' ``the real result of this analysis was these three plots, so go back and figure out everything that I did that was involved in creating those three plots. So that I could start from the raw data and create those three plots.''


\subsection{Long-running cells}
Kenny: Problem of long-running cells.  ``The long tail.''  First cell takes seconds to run, next cell minutes, next cell 5 hours.  In this scenario, which is common for Kenny, the ``Run whole notebook'' button is dangerous. This is also a huge issue for demoing code.  Suggests something like knitr's caching of results. When converting his work to notebooks, Kenny ends up with a lot of comments that say ``this cell takes a long time to run.'' He ends up ``littering the notebook with little switches that comment out'' the slow parts and ``load the object instead'', ``save the object to disk''.


\subsection{Misc. notebook interface annoyance}
Rick: ``if I was typing into one of the cells near that top, I had to think of `I'm editing this cell and then I have to execute it', and if was doing the one at the bottom I could type it but then it would automatically execute, but then it became a standard cell and I have to edit it. It was just another set of modes that I didn't like''

Clint: ``it took me a couple of weeks of looking at it to become comfortable because I saw the cells and it just threw me for a total loop. I mean, it's a good idea because it's essentially like I can have this part and this part and this part. I can run it all at once if I want or I can break these into sections for either debugging or staging purposes which, I really like it now, but when I first saw it, [it was] very confusing... I think I prefer it now, because you can cut your code up.''

Corey: In the previous version, it switched between code and results. Now you see both - ``and there are times when that's good, but there's certainly times when it'd be nice to - you know, you have 10 or 15 cells and you just want to get back and there's all this output and you're scrolling all over the place so, if there were a way to control both, that'd be great. There are times when I do want to see the output, but there are times when I've seen enough of the same thing.''

Jyothi: ``It's good that I can see the results but if I'm working and I need more room and I want to hide the results, I can't.” Could we have “an option to close the results window?”


Ivan: RStudio's layout is more helpful because
- the charts are kept in another pane and stay in one place while doing analysis
- having ``long term'' code (which would be in assets) above the command prompt means that both can be sized very wide to accomodate long lines

Kenny: Problem with screen real estate: side panes take up too much space. Doesn't need the notebook tree pane.

KC thought that having a scratch pad that doesn't persist between different notebooks is 
``useless''.


\subsection{Other workflows}
Rick: ``I find that the style of editing is clunky. I would never edit like that if I'm doing ordinary work. I usually use vi.'' ``I mean it's nice to have it saved, but you know, there's sort of this trade-off between it making it easier for me to present something or to save something, and my easy of typing and correcting and things in a plain old editor window.'' ``I just have the discipline of having a file that describes what I'm doing with the commands that I'm using so that I can go back and recreate it or pass it on to someone else, so it's a little bit like having an RCloud notebook, it's not necessarily executable but most of the commands that I've typed are in there.''

Kenny: the UI doesn't work for his workflow, which is pasting snippets from a file into the console in the R GUI.  he tends to type type type into the file, paste paste paste to the console, rearrange so the good code is at the top of the file, then save the file and call it quits once he's got the right code at the top of the file.

Taylor: ``The big one is I don't know why would I want to use RCloud over my current setup. If it's just me, I like my text editor and terminal. There's nothing that I want that those two don't give me'' ``In R, when I'm doing development work, I copy and paste in like 5-10 lines of code, so when something breaks, I get an error message on that one line, and I can up-arrow and change it and fix it, whereas in RCloud I have to run a whole cell, so the only way to get that same functionality is if every line's in one cell. Because if it runs like lines 1,2,3, then 4 fails, I have to either run the whole cell again, or have line 4 in its own cell. I can't rerun just one line.'' ``There's nothing that I want that's not in the terminal. I don't think you should try to support all the things I want to do in the terminal.''

Byers: installing random packages is part of exploratory data analysis


\subsection{Version management difficulties}
Rick: ``I don't need something that keeps track of every mistake I've made or every direction I've tried.” ``namespaces are already crowded, so trying to remember the names of notebooks is hard, much less the names of states within each notebook.''


\subsection{A sea of notebooks}

Byers: Problem with forking a notebook and correcting it, is that the original notebook still exists with the error.  We need ways to see that there is a more recent fork of notebook and to fetch in the changes from it.

Taylor: ``I'll make a notebook of a little tutorial of, here's the basic functions and what they do, and then I'll send them out, and then people will give me feedback on them, or have some new feature, or I'll realize something's wrong, and then I'll go back and change it and update that page, but then what happens is people have in the meantime, forked it, and then they have a version that… and usually what it'll be is actually that I'll have to change a package as well, so their old fork stops working, and then they complain, and then, oh no, you have to go get the original [notebook]. Yeah, that's the sort of.. This probably bites me more than other people, because I'm doing development so things are changing as they're looking at the code. Like, iotools, a lot of the functions change names or change syntax.''

Taylor: ``That's just a problem because people will have old notebooks that don't work anymore. It's annoying when it's a notebook that I still have and the format has changed just a bit. More than that is when people have, now that I know that you can't even delete notebooks, they'll have a fork of something that doesn't exist anymore on my thing.'' ``Now I'm really gun-shy about sharing notebooks because, I don't want to like, it's like, do I want to support this forever?''

Taylor: ``A good example is, someone was complaining about doing forking in an RCloud notebook. They were trying to use a parallel package. And I said, no no I think it works, and I did like a one-cell thing that just did some sweep kind of thing to show that it was actually working in parallel, and then I just deleted it, thinking it was gone. And then it saved something in a directory that I then deleted, because it was just a temporary directory. Three months later, I don't know why they were running it, they said `oh it doesn't work'. I think that's the biggest issue I'm having right now, all these forks''

Taylor: ``there's no way for both of us to have ownership of a notebook, so the only way is to fork it back and forth, and so we have dozens of old copies. [We end up] deleting all the old ones, but people still have links to them, because they don't actually disappear.'' ``The UI makes it easy to go back to the old stale versions, because you can say `forked from' `forked from' `forked from', but there's actually no way to go the way that we want people to go, which is, like, up the tree.''

Rick: ``I don't necessarily need to see everybody's notebook that uses RCloud'' ``every time I do something new, I get a new notebook and so now my notebooks are maybe 50 or 60. That's enough to think about just on my own, but if I've got everybody's 50 or 60 sitting on my display, I find that it's more than I want to know about.''


\subsection{Parameterization}
Clint: ``For the things that I'll do multiple times, I'll try to parameterize it, and I'm working now on putting some packages together that will operate in some fashion outside of RCloud, straight from the command line''

Taylor: ``like the top of it would be, `This is a report of the volume of all of our feeds for this month', and someone would want to look at it for the next month or the previous month, so they'd fork it to change the month''

Jyothi: ``I think that if a notebook is web-enabled [parameterized] then it gives the person who is running the notebook, I mean they can always fork the notebooks and make changes, but I feel that if the owner of the notebook web-enables it, then whoever is running it, it's easier to run, instead of forking it, if they can set options, it's probably more efficient, and also they can fork it if they want to.''


\subsection{Web snafus}
Disconnection, slowness

Corey: ``“What frustrates me is that if I do leave this window for what feels like 5 or 7 minutes, it disconnects and I have to reconnect and run the whole notebook again.''

Kenny: it really sucks when you get disconnected

Raif: ``being web-based, it's obviously slower than something you would be running on your machine''

Byers: even the little lag of sending a command to a server and receiving the result is intolerable


\subsection{Other}
Byers: There is also a big problem with ``bitrot'' - notebooks often stop working because the user changed the structure of their data, changed a filename or a database.  We need ``organizational protocols'' to catch up with the technology.  There is a ``hysteresis between the technical and the social'' and we have not yet adapted.

Taylor: `` It would be a file inside a directory that only I had read access to and so someone else couldn't run it... That was usually what it was, it was a temporary file on something that was read-only to everyone else. But I've gotten careful about that.''

Bart: ``One UI notebook and a bunch of other notebooks that do various tasks and are called with API calls. The problem is, if you clone all the notebooks, then you have to go update every single notebook, because all the IDs are different now. There's no sort of relative path type stuff that you can do. You can do it with static assets, but you can't do it with Ajax calls''

\section{Lessons Learned, Discussion and Limitations}

What are the decisions and ideas that are central to this paper?

% limitations
Previous studies have pointed out difficulties in achieving the flexibility,
scalability and maintainability expected of production software in experimental
code written by data analysts.
Although these properties cannot be enforced by any programming environment,
suitable tools can help well-motivated analysts and programmers to create
robust applications with much less effort.

Asynchronous collaborative visual analytics
(ACVA)~\cite{Chen:2011:SEC}. This paper addresses visualization
\emph{of} the ACVA process. It will be important when we talk about
recommendation systems, and navigation of the set of notebooks, etc.

Had to back off only one language sooner than we expected.

Returning to the design considerations outlined by Heer and Agrawala, experience with the RCloud prototype demonstrates the value of several
Shared artifacts, artifact histories,
Discussion
View sharing - bookmarking, everything available as a URL
Content export - want to avoid, create ways of working where this is not necessary.
Social-psychological incentives - starring
Voting and ranking - starring

Group management, size, diversity.  Lacking support,but clearly desirable.

Has to play well in the ecosystem of the other tools and worked out as planned but there is a temptation to absorb

discussion about binding times, running vs.caching.
transparent vs. explicit operations done by programmers

What is relationship to a standard CMS with R/Python

Elaborate on to what extent can the users of the production notebooks use the tools like annotation etc. without coding or without being exposed to the analysts view. is this a gap in our design.

Shared Artifacts
Artifact Histories
Activity Histories


Discussion
Recommendation
Social-Psychological Incentives
Voting and Ranking


View Sharing / Bookmarking
Content Export
Presentation

\section{Conclusions and Future Work}

We presented a case for an environment that supports the visual
analytics process, from data acquisition and exploratory
data analysis, through development and deployment. One might
think of this approach as ``DevOps for data science.''
The environment supports collaboration and communication
with capabilities for creating and sharing experimental workbooks,
version control, searching, recommending, annotating, and publishing
experiments as web sites and as reusable services.

We implemented these ideas in the RCloud prototype.

Experience with the prototype provides evidence that data
science teams, and the organizations in which they work,
benefit from having such capabilities.

Some possible next steps are to support cross-language development,
to incorporate richer source code analysis and recommendation
techniques, to provide fine-grained security, and to improve
the usability of the HTML5 human interface.

The code is available 
at \url{github.com/att/rcloud/}
under an MIT open source license.


\bibliographystyle{abbrv}

\bibliography{paper}
\end{document}
