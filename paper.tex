%\documentclass[journal]{vgtc}                % final (journal style)
\documentclass[review,journal]{vgtc}         % review (journal style)
%\documentclass[widereview]{vgtc}             % wide-spaced review
%\documentclass[preprint,journal]{vgtc}       % preprint (journal style)
%\documentclass[electronic,journal]{vgtc}     % electronic version, journal

%% Uncomment one of the lines above depending on where your paper is
%% in the conference process. ``review'' and ``widereview'' are for review
%% submission, ``preprint'' is for pre-publication, and the final version
%% doesn't use a specific qualifier. Further, ``electronic'' includes
%% hyperreferences for more convenient online viewing.

%% Please use one of the ``review'' options in combination with the
%% assigned online id (see below) ONLY if your paper uses a double blind
%% review process. Some conferences, like IEEE Vis and InfoVis, have NOT
%% in the past.

%% Please note that the use of figures other than the optional teaser is not permitted on the first page
%% of the journal version.  Figures should begin on the second page and be
%% in CMYK or Grey scale format, otherwise, colour shifting may occur
%% during the printing process.  Papers submitted with figures other than the optional teaser on the
%% first page will be refused.

%% These three lines bring in essential packages: ``mathptmx'' for Type 1
%% typefaces, ``graphicx'' for inclusion of EPS figures. and ``times''
%% for proper handling of the times font family.

\usepackage{mathptmx}
\usepackage{graphicx}
\usepackage{times}
\usepackage{xspace}
\usepackage{amssymb}

% \usepackage[onecolumn]{multicol}

%% algorithm packages (llins)
\usepackage{algorithm}
\usepackage{microtype}
\usepackage{algpseudocode}

%% We encourage the use of mathptmx for consistent usage of times font
%% throughout the proceedings. However, if you encounter conflicts
%% with other math-related packages, you may want to disable it.

%% This turns references into clickable hyperlinks.
\usepackage[bookmarks,backref=true,linkcolor=black]{hyperref} %,colorlinks
\hypersetup{
  pdfauthor = {},
  pdftitle = {},
  pdfsubject = {},
  pdfkeywords = {},
  colorlinks=true,
  linkcolor= black,
  citecolor= black,
  pageanchor=true,
  urlcolor = black,
  plainpages = false,
  linktocpage
}

%% If you are submitting a paper to a conference for review with a double
%% blind reviewing process, please replace the value ``0'' below with your
%% OnlineID. Otherwise, you may safely leave it at ``0''.
\onlineid{}

%% declare the category of your paper, only shown in review mode
\vgtccategory{Research}

%% allow for this line if you want the electronic option to work properly
\vgtcinsertpkg

%% In preprint mode you may define your own headline.
%\preprinttext{To appear in an IEEE VGTC sponsored conference.}

%% Paper title.

\title{RCloud: Integrating Exploratory Visualization, Analysis and Deployment}
%% \title{RCloud: DevOps for Visual Analysis and Data Science}

%% This is how authors are specified in the journal style

%% indicate IEEE Member or Student Member in form indicated below
\author{Stephen North and Carlos Scheidegger and Simon Urbanek and
  Gordon Woodhull}

\authorfooter{
%% insert punctuation at end of each item
}

%other entries to be set up for journal
% \shortauthortitle{Biv \MakeLowercase{\textit{et al.}}: Global Illumination for Fun and Profit}
%\shortauthortitle{Firstauthor \MakeLowercase{\textit{et al.}}: Paper Title}

%% Abstract section.
\abstract{ Consider the emerging role of a data science team within an
  organization. Individual data scientists and statisticians usually
  work on loosely related problems, and must devise effective ways to
  share their findings and move results from exploratory data analysis
  into automated diagnostics and reports deployed for wider consumption.
  There are two problems with the current
  practice. First, there are gaps in this workflow: EDA is performed
  with one set of tools, and automated reports and deployments with another.
  Second, EDA environments often assume a single-developer perspective,
  while data scientist teams would get much benefit from easier sharing
  of scripts and data feeds, experiments, annotations, and automated
  recommendations, which are well beyond what traditional version control
  systems provide. We contribute and justify the following three
  requirements for systems built to support current data science teams
  and users: \emph{technology transfer}, \emph{coexistence}, and
  \emph{discoverability}. In addition, we contribute the design and
  implementation of RCloud, a prototype system that supports the requirements
  collaborative data analysis, visualization and web deployment. We
  discuss the design decisions, tradeoffs and limitations, and compare RCloud
  to other current proposals.
} % end of abstract

%% Keywords that describe your work. Will show as 'Index Terms' in journal
%% please capitalize first letter and insert punctuation after last keyword
\keywords{Some, Keywords, Here}

%% \CCScatlist{ % not used in journal version
%%  \CCScat{K.6.1}{Management of Computing and Information Systems}%
%% {Project and People Management}{Life Cycle};
%%  \CCScat{K.7.m}{The Computing Profession}{Miscellaneous}{Ethics}
%% }

%% Uncomment below to include a teaser figure.
\teaser{
  \centering
  \includegraphics[width=.3\linewidth]{fig/screenshots/wdcplotsmall.jpg}
  %% \includegraphics[width=.9\linewidth]{figs/teaser.jpg}
  % \vspace{-.5em}
  \caption{RCloud provides infrastructure for data science teams so
    that R \emph{hackers} and \emph{scripters}~\cite{Kandel:2012:EDA} can target modern
    interactive visualization technologies such as
    D3~\cite{Bostock:2011:DDD}, dc.js~\cite{DCJS:2014:DC} and
    Crossfilter~\cite{Square:2014:CFM} (left), while at the same time
    enabling seamless deployment and dashboarding for
    \emph{application users} (right)
  }
% \vspace{-.5em}
}

%% Uncomment below to disable the manuscript note
%\renewcommand{\manuscriptnotetxt}{}

%% Copyright space is enabled by default as required by guidelines.
%% It is disabled by the 'review' option or via the following command:
% \nocopyrightspace

%
% Useful Macros
%

\newcommand{\eg}{e.g.\xspace} % e.g. and i.e. ARE NOT ITALIC!!
\newcommand{\ie}{i.e.\xspace}
\newcommand{\todo}[1]{\textcolor{red}{#1}}

\newcommand{\jim}[1]{{\color{green} Jim: [{#1}]}}
\newcommand{\carlos}[1]{{\color{blue} Carlos: [{#1}]}}
\newcommand{\lauro}[1]{{\color{red} Lauro: [{#1}]}}


%%%%%%%%%%%%%%%%%%%%%%%%%%%%%%%%%%%%%%%%%%%%%%%%%%%%%%%
%%%%%%%%%%%%%%%%%%%%%% ALGORITHM %%%%%%%%%%%%%%%%%%%%%%
%%%%%%%%%%%%%%%%%%%%%%%%%%%%%%%%%%%%%%%%%%%%%%%%%%%%%%%

\renewcommand{\algorithmicthen}{}

% aux. commands

%%%%%%%%%%%%%%%%%%%%%%%%%%%%%%%%%%%%%%%%%%%%%%%%%%%%%%%%%%%%%%%%
%%%%%%%%%%%%%%%%%%%%%% START OF THE PAPER %%%%%%%%%%%%%%%%%%%%%%
%%%%%%%%%%%%%%%%%%%%%%%%%%%%%%%%%%%%%%%%%%%%%%%%%%%%%%%%%%%%%%%%%

\begin{document}

%% The ``\maketitle'' command must be the first command after the
%% ``\begin{document}'' command. It prepares and prints the title block.

%% the only exception to this rule is the \firstsection command


\firstsection{Introduction}

\maketitle

\todo{Points we want to make."Coordination is done by meetings" "how do you trace an automated alert back to an EDA environment ?"}

% What's the area
Consider the emerging role of a data science team within an
organization today \cite{Keim:2008:VAS}.

% How do people do it today?
Individual data scientists and statisticians usually work on loosely
related problems, and must devise effective ways to share their findings
and move results from exploratory data analysis to automated diagnostics
and reports that are deployed for wider consumption.
They generally rely on a patchwork of resources shared informally,
or at best, through systems that attack individual tasks or functions
in the data analysis workcycle. % might cite information visualization reference model or pirolli/card

% Why does this not work?
There are two problems with the current
practice. First, there are gaps in the workflow:
Exploratory Data Analysis (EDA) is done with one set of tools,
and automated reports and deployments with another. This makes
transferring results from EDA to production slow, expensive, and error-prone.
Second, EDA environments often assume a single-developer perspective,
while data scientist teams could greatly benefit from convenient sharing of
scripts, data sets, data feeds, experiments, annotations, and automated
analyses and recommendations. These features are well beyond what existing
repositories and version control systems provide.

% What are we going to do about this?
\todo{Sell idea, not system} In this paper, we investigate the hypothesis
that the practice of visual analytics can advance by adopting techniques
from information retrieval and collaborative work environments. To investigate
these ideas, we designed and deployed a prototype environment called RCloud,
that supports collaborative data analysis, visualization and web deployment.
We will report on the design decisions, tradeoffs and limitations, and compare
the concepts in RCloud to similar proposals.

\todo{Define the requirements for the system}

We begin by noting the main requirements for a data analysis environment for
collaboration and transfer of results into production.

\begin{itemize}
\item Technology transfer. Experiments in notebooks developed by analysts need to become accessible in the production environment, making it easy to push new features, prototypes, etc.
\item Dashboarding. To support the above, there should be a mode in which non-experts
can operate the experiments without coding.
\item Version control as a primary concept in the repository. There needs to be a way to control the visibility of prototypes as they are being developed and tested.
\item Modern interactive technologies. People want to interact with data to explore it. Scalable technologies like crossfilter exist. Want to support those in the context of an R environment. Need to support full 2-way communication between computing in the cloud and interaction on the web.
\item Other requirements e.g. Kandel's user study. Not our contribution but things we must address, mentioned in related work section. Heer/Agrawala design considerations for collaborative visual analytics.
\end{itemize}

% What is the scope of our solution?
Our investigation focuses on the R language \cite{RCoreTeam:2013:R}
for statistical computing and graphics. R is one of the most popular
systems for ...

% We want our solution to play well with the rest of the ecosystem
R has developed a complex ecosystem of compatible packages, software
tools and other resources. For example there are packages that make
it convenient to execute map-reduce programs, generate web interfaces,


\section{Related Work}

Kandel et al. argue in their review of data wrangling work that data
cleaning, wrangling and transformation is a major part of exploratory
analysis and visualization~\cite{Kandel:2011:RDI}. Although RCloud
does not by itself include modules specific to data cleaning and
wrangling, we note from internal experience that these cleaning
scripts themselves tend to change over time. RCloud addresses this
issue by providing easy publishing of data-cleaning notebooks as web
services, which reduces the total data-cleaning effort across an
organization.

Kandel et al.'s interview study points out the typical ``explore'',
``model'', ``report'' cycle in enterprise data
analysis~\cite{Kandel:2012:EDA}. There are many discontinuities in
this cycle that cost time and effort to overcome. RCloud seeks to
reduce this impedance mismatch. They also point out that larger teams
are becoming more common in data analysis, that supporting
collaboration is a difficult and important problem, and that sharing
and versioning of data sources and artifacts is hindered by current
technology in practice. ``We found that analysts typically did not
share scripts with each other. Scripts that were shared were
disseminated similarly to intermediate data: either through shared
drives or email. Analysts rarely stored their analytic code in source
control.'' Their work points to the opportunity for better technology
to support collaboration and sharing by data analysis teams.

Heer and Agarwala identify many design considerations for
collaborative visual analytics~\cite{Heer:2008:DCF}. \emph{Starring},
the means for signaling interest in notebooks, described in
Section~\ref{sec:starring}, addresses social-psychological incentives,
recommendation, and voting and ranking. RCloud's integrated deployment
mechanism, described in Section~\ref{sec:deployment}, addresses cost of
integration, content export, presentation and view sharing. Notebooks,
and the integrated version control system for them, described in
Section~\ref{sec:notebooks}, address modularity and granularity, and
artifact histories.

Manyeyes \cite{Viegas:2007:MAS} was a landmark system for the integration
of social media with visualization and data publishing. We build on this
work by defining a rich interface for collaboration about code and for
operating on metadata.

The need for integrating statistics and visualization has been
highlighted in previous studies and is widely understood by
various technical communities \cite{Perer:2008:ISA}

There has been noteworthy work on specific techniques such as
social bookmarking \cite{Millen:2006:DSB} \cite{Heer:2007:VAV}
and crowdsourcing \cite{Fast:2014:ECS} to support collaborative
or social development or analysis processes.
Similarly, there are computational methods to support high
performance execution in incremental code development
environments \cite{Guo:2010:TPI}.
Our goal is to define an environment in which many such
techniques may be integrated and made available to a broad community.

VisMashup~\cite{Santos:2009:VST} and Crowdlabs~\cite{Mates:2011:CSA}
are tools built on top of the VisTrails workflow management
system~\cite{Callahan:2006:VVM}. VisMashup defines a schema and
semantics for automatically deriving user interfaces from workflows,
while Crowdlabs exposes these capabilities on a website feature
workflow upload and remote execution. In our view, the impedance
mismatch between a dataflow pipeline specification and the power of a
general-purpose language is too great for the type of general
exploratory work in data science teams. As a result, RCloud tries to
provide a closer data-analysis to analysts accustomed to creating and
executing R and Python code while keeping many of the attractive
properties like automatic versioning and management.

Our work has been inspired by and benefited from other proposals
to improve data analysis environments and processes,
including RStudio \cite{RStudio:2013:SWA},
R packages such as Markdown \cite{Allaire:2014:MMR},
knitr \cite{Xie:2013:DDW}
and Shiny \cite{RStudio:2013:SWA},
and iPython notebooks \cite{Perez:2007:IAS}
to name a few. RStudio aims at providing an integrated development environment
for R programming, with support for publishing code in packages. Markdown,
knitR and Shiny provide R with sophisticated reporting capabilities, including
interactive web interfaces. IPython \cite{Perez:2007:IAS}
shares many of our goals, such as providing a comprehensive environment
for analysis and programming, with shareable documents on the web.

One overall goal is to reduce the gap between implementers and deployers
of technology. Don't want a team of 20 IT people to support 5 data scientists.

% devops for data science.

\section{The System\label{sec:system}}

\subsection{High-level Architecture\label{sec:highlevelarchitecture}}

% Stephen: Let's work from front to back, from human interface to the
% services and frameworks that support it.

The internal computing infrastructure of organizations has changed
radically in the last fifteen years. The shift from large
servers toward scalable, lower-cost, distributed systems (``the cloud'')
led to a software ecosystem of processes distributed over a
network, usually communicating via HTTP. HTTP is dominant because
web browsers and servers are ubiquitous and available in almost
all hardware devices, from tiny sensors, to handheld devices and
laptops, to rack-mounted servers.

As a result, HTTP is the lingua franca of interprocess communication
(IPC). One of the design goals for RCloud was to provide a productive
environment for creators of data-analysis scripts, that also behaves
as a first-class citizen in the pre-existing ecosystem of computer
services and networks in an organization.

%% Design for cloud-friendly? This goes back to the point in the
%% introduction about playing nice with the rest of the ecosystem.

\subsection{Human Interface\label{sec:humaninterface}}

%\stephen{What??}
%We designed RCloud around the front-facing API, which provides roughly
%one entry point to correspond with each requirement.

Figure~\ref{fig:something} shows the RCloud developer interface.
The center panel is a notebook for code and visualizations.
Code is edited in this panel, and visualizations are rendered.
An inventory of supplementary ``asset'' files, that are part of
the notebook, though not in its executable flow, are edited on the right.
Controls at the top of the screen allow the user to run, view, and
share the notebook. The result of code execution is the final cell of
the notebook.  On the left is a browsing area for searching and viewing
other users' notebooks, and a help system.

The call mechanism is the URL itself; arguments are read from the URL
and the final cell is rendered as the result. 

\subsection{Notebooks\label{sec:notebooks}}

The main unit of computation in RCloud is a \emph{notebook}.
A notebook holds a sequence of \emph{cells}, each of which contains a
snippet of code or hypertext in Markdown. This is not a novel idea;
executable documents structured this way are a feature of many
other systems, including Mathematica, IPython and Sage.
\stephen{Code is executed when...?}

One of the main contributions of RCloud is the idea that notebooks
are ``always deployed''. In other words, the most recent version of
a notebook is immediately available.\stephen{What if it is broken?}
This makes it convenient to share and modify experiments and compose
results without binding to a specific version of a notebook.
On the other hand, there are situations where it might be important
to name specific versions. We do not expect designers to decide which
versions need to be preserved, but we embrace \emph{transparent} versioning.
This is similar to models like Jankun-Kelly et al.'s p-set calculus \cite{Jankun-Kelly:2007:MFV}
and VisTrails's version tree \cite{Callahan:2006:VVM}, where every change in the state of the system is tracked.

To implement this, we built RCloud on top of Github's \emph{gists}~\cite{Github:2014:GG}.
Github offers a HTTP interface for creating simplified git repositories, the main limitation
being a restriction to text-only files in a single directory. The GitHub web-service
API provides most of the semantics we need for the versioning portion of the storage back end:
access to previous versions, comments, starring, and forking.

Using GitHub for storage and versioning also exposes other capabilities
that can be invoked with JSON and HTTP.
Particularly, we can provide full text search using Apache SOLR.
SOLR is scalable, can index in near-realtime, supports multiple character sets,
indexes several common types of documents, and has schemas and faceted search.
%
Integrating existing services and software components, especially open source,
instead of implementing custom code is a trend in the software industry, and
is very positive for systems research and prototyping in visual analytics.
By adopting existing technology, small teams with limited resoures can explore
novel ideas toward the improvement of large, complex software ecosystems.
In our case, even though this approach involved learning standards not directly
related to visual analytics, our early decision to use the GitHub gist API
turned out to be successful. Future projects could gain many of the same benefits
by adopting this strategy. In fact, it will be essential for technology adoption
and transfer, because it is almost impossible for any one tool or platform to
``own'' or support the entire visual analytics process.
\stephen{Are we preaching too much?}

\subsection{Reputation and Interest: starring\label{sec:starring}}

Information retrieval based on collecting usage and recommendations
is a cornerstone of modern web services. We would like to help data
scientists to find workbooks (and therefore items in their contents
such as code, data, and colleagues with specific expertise) with
the benefit of such information.

In RCloud, reputation and interest are a relationship between
\emph{notebooks} and \emph{users}, rather than a relationship between
user pairs. We chose this approach because we expect initial 
RCloud deployments to have relatively few users, but some users to
create many notebooks. Under that assumption, assigning interest
to users would not provide sufficiently ``high-resolution'' data.

We incorporate both explicit and implicit indications of interest
in notebooks. Explicit interest is indicated by ``starring,'' or
clicking on a button that marks a notebook as interesting.
This makes explicit indication of interest a nearly trivial operation,
always readily available, to encourage its use.

Implicit signaling of interest is supported by keeping click-through counts
\cite{Joachims:2005:AIC} and execution counts. (In addition to these
standard techniques of collecting feedback from web search, we anticipate
applying static and dynamic code analysis to infer fine-grained
information about relationships, for example, which packages and data
sets often appear together.)

\subsection{Executing R in a web browser\label{sec:Rinbrowser}}

One our main goals is to provide ubiquitous access to the statistical
tools of the R programming language in the wider environment in which data
science teams participate.
%
Given the recent developments of interactive visualization and visual
analytics in HTML5 and the web, we decided to create an
\emph{interconnect} between the two environments.
%
Every RCloud session spawns a new R execution process in a remote
server.

%% \subsubsection{Object capabilities: R to Javascript RPC and access
%%   control}

HTTP and web services are convenient but much of the protocol is
stateless. (For example, GET requests are required to not change the
remote state, and results can be cached in transparent proxies along
the network). In the case of close communication between a running
R process and a web browser, we need something that is more flexible
than distinct URLs for every R process, and which embraces
statefulness of both calculations and visualization parameters.

\carlos{In-depth explanation of R<->web interconnect, and why it's
  necessary or important. Main point: make programming in the R side
  as close as possible to what R programming feels like, and make
  programming in the Javascript side as close as possible to what
  Javascript programming feels like. We do this because this part of
  the system is targeted at programmers; Shiny targets web programming
  at non-web-programmers. We don't attempt such a thing and land at a
  different point in the spectrum.}

\carlos{compare against https://www.opencpu.org, compare against
  IPython notebooks. Two-way communication? Actually describe what it
  is. IPython manipulate notebooks? Understand and contrast}

\subsection{Interactive notebooks\label{sec:interactivenotebooks}}

%% How do we do things that are not trivial to do with IPython (for
%% example)

%% What is relationship to a standard CMS with R/Python

%% dcplot. two-way communication between between backend session and
%% frontend session.

A key engineering decision in RCloud was to rely on full two-way
communication between the web client human interface, and data and
computing resources in the cloud. It is easier to engineer a design
in which the human interface is pushed one-way from the server
and thereafter all interaction takes place within the client.
Two-way communication is necessary, though, where the size of
the data or computational performance makes it impractical to
move all computation to the client.

In addition, R is well suited as a language for analysis, and
JavaScript for interactive visualization. To draw on
the strengths of both languages and both environments, the connection
between the languages is not just procedural: it makes \emph{closures}
and \emph{first-class functions} available across the network.
This provides considerable flexibility, so that for example, a chart built
with dc.js or leaflet.js can call back to analysis functions in R
without having to formalize the protocol between the processes.

\subsection{Deployment of notebooks\label{sec:deployment}}

Every notebook in RCloud is named by a URL, and notebooks by
default are visible by the entire organization. This is deliberate.
As pointed out by Wattenberg and Kriss~\cite{Wattenberg:2011:DFS},
broad access to analysis outputs (in their case, for NameVoyager)
increases long-term engagement in part through cross-references on
the web. Although our prototype RCloud deployment is only visible
inside a corporate intranet, we nevertheless found anecdotal support
for this notion by discovering links to RCloud notebooks in internal
discussion fora and mailing lists.

Because notebooks are continually published, any work is immediately
available both as a subroutine and a visual component. Close colleagues
can start on the next stage of analysis, or delve into the data,
even while the original author is polishing an algorithm or its
presentation. \stephen{We glossed over the things that can go wrong.
We should at least discuss that briefly.}
The code is the page, and can either be shaped into
a function of inputs and outputs, or the linear cells of the notebook
can be reworked into a full-fledged HTML layout.

\section{Case Studies}

In this section we present a pair of example applications of the
capabilities provided by RCloud. The examples themselves are synthetic
(unfortunately, we are unable to show production notebooks because of
their business value), but the examples are representative of the
type of progression we want the ecosystem to support.

\subsection{Stock Price Analysis\label{sec:stockvis}}

\begin{figure}
\includegraphics[width=\linewidth]{fig/casestudy1/casestudy1.pdf}
\caption{\label{sec:stockvis}Iterations of a stock ticker
  visualization, based on an example suggested by Hadley Wickham.  A
  simple, static visualization of the closing price of a single ticker
  is progressively improved into a configurable visualization suitable
  for dashboarding, then into an interactive visualization of the
  volatility and volume of two separate stocks, then finally into API
  calls for data access use in other RCloud notebooks. All notebooks
  in this can be used directly as web pages. When a notebook
  corresponding to a function call is used as a web page, its included
  documentation is displayed.}
\end{figure}

Our first example is a sequence of visualizations of the performance
of stock prices over a multi-year period. The first visualization uses
ggplot2 and is due to Hadley Wickham. It uses data provided by Yahoo!
Finance via their web service, and shows the stock price for a single
symbol.

The webpage that produces that visualization always includes a link
back to the generating source code, from which a user can always fork
the notebook and, in this case, add a configurable ticker based on the
URL of the notebook. Notice how the notebook changes very little.

From there, the notebook author (or a user) can decide to provide easy
access to the data without producing a visualization. This is achieved
by simply creating a notebook that defines a function. This notebook
then becomes a \emph{subroutine} for other notebooks, and is
version-controlled in the same way.

Interactive version.

%% This notebook can then be called in an \emph{interactive}
%% visualization, which uses the Javascript 

%% Easy to convert it into configurable entry point for dashboarding.

%% Now consider an analyst that wants to understand the price dynamics of
%% different stock attributes, different points in time, etc. For that,
%% interactive visualization is very attractive. describe dcplot and
%% notebook which uses it.

%% Then describe abstraction of stock-data fetching, to talk about
%% calling notebooks from other notebooks.

%% then call.r?

\subsection{text analysis\ref{sec:textvis}}

Same.

% IMPORTANT: what is unique about RCloud here
% From prototyping to dashboard
% Getting other information from the web
%

\section{Lessons Learned, Discussion and Limitations}

What are the decisions and ideas that are central to this paper?

% limitations
Previous studies have pointed out difficulties in achieving the flexibility,
scalability and maintainability expected of production software in experimental
code written by data analysts.
Although these properties cannot be enforced by any programming environment,
suitable tools can help well-motivated analysts and programmers to create
robust applications with much less effort.

Asynchronous collaborative visual analytics
(ACVA)~\cite{Chen:2011:SEC}. This paper addresses visualization
\emph{of} the ACVA process. It will be important when we talk about
recommendation systems, and navigation of the set of notebooks, etc.

Had to back off only one language sooner than we expected.

Returning to the design considerations outlined by Heer and Agrawala, experience with the RCloud prototype demonstrates the value of several
Shared artifacts, artifact histories,
Discussion
View sharing - bookmarking, everything available as a URL
Content export - want to avoid, create ways of working where this is not necessary.
Social-psychological incentives - starring
Voting and ranking - starring

Group management, size, diversity.  Lacking support,but clearly desirable.

Has to play well in the ecosystem of the other tools and worked out as planned but there is a temptation to absorb

discussion about binding times, running vs.caching.
transparent vs. explicit operations done by programmers

What is relationship to a standard CMS with R/Python

Elaborate on to what extent can the users of the production notebooks use the tools like annotation etc. without coding or without being exposed to the analysts view. is this a gap in our design.

Shared Artifacts
Artifact Histories
Activity Histories


Discussion
Recommendation
Social-Psychological Incentives
Voting and Ranking


View Sharing / Bookmarking
Content Export
Presentation

\section{Conclusions and Future Work}

We presented a case for an environment that supports the visual
analytics process, from data acquisition and exploratory
data analysis, through development and deployment. One might
think of this approach as ``DevOps for data science.''
The environment supports collaboration and communication
with capabilities for creating and sharing experimental workbooks,
version control, searching, recommending, annotating, and publishing
experiments as web sites and as reusable services.

We implemented these ideas in the RCloud prototype.

Experience with the prototype provides evidence that data
science teams, and the organizations in which they work,
benefit from having such capabilities.

Some possible next steps are to support cross-language development,
to incorporate richer source code analysis and recommendation
techniques, to provide fine-grained security, and to improve
the usability of the HTML5 human interface.

The code is available 
at \url{github.com/att/rcloud/}
under an MIT open source license.


\bibliographystyle{abbrv}

\bibliography{paper}
\end{document}
