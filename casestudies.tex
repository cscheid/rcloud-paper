\section{Case Study\label{sec:casestudy}: visual exploration of topic modeling in RCloud}

To illustrate ...
As an example, we present a simple real-world application
deployed under RCloud.
The application is an implementation of LDAVis,
a method for visualizing large text corpora.
LDAVis helps non-experts to explore collections of
short text documents using topic modeling and
visualization. Topic modeling~\cite{Blei:2003:LDA}, although
powerful, often requires human interaction
and interpretation~\cite{Sievert:2014:LAM}.

\begin{figure*}
  \includegraphics[width=\linewidth]{fig/casestudytext/casestudytext.pdf}
  \caption{\label{fig:textvis}An example application developed and deployed in RCloud.}
\end{figure*}

LDAVis was developed by two technical staff members at
AT\&T Labs, and originally was written in RStudio's
Shiny~\cite{RStudio:2013:SWA}, a framework for developing R
applications for the web.  While Shiny provides outstanding ease of
development, it turns out that discoverability and deployment are equally
important aspects in the lifecycle of an internal application (see
Section~\ref{sec:interviews}). In these aspects, RCloud provides a
simple solution: \emph{all developed notebooks are automatically deployed}.

LDAVis combines text analysis, dimensionality reduction and
interactive visualization. Text analysis is performed by
combining a standard R library to fit LDA models using
Gibbs sampling~\cite{} with custom R code written by the
developers. The analysis module is a single R function
exposed to the web application via the Javascript-R RPC
mechanism described in Section~\ref{sec:system};
thus analysis is performed remotely on an RCloud server.

Each text topic is a probability distribution over the all the
words of the document. To expose patterns in the relationships
between topics, LDAVis employs a combination of
interactive visualization and dimensionality reduction,
allowing users to adjust measures for topic distances and the choice
of dimensionality reduction technique. The dimensionality reduction
algorithms and distance measures are implemented in R, which
means they can be executed on the RCloud servers as well.

The result of the dimensionality reduction process is a
two-dimensional plot of the topic space, an example of which is shown in
Figure~\ref{fig:textvis}. The interactive view is implemented in SVG
and Javascript through D3~\cite{Bostock:2011:DDD}. One of the most
popular web-based visualization libraries for R is ggvis~\cite{ggvis},
so it is natural to ask if ggvis could have been used instead of
custom Javascript. In this case, the interactive features of ggvis
(and, by extension, Shiny) are a subset of those in Vega~\cite{vega}.
Custom interactions in LDAVis (such as hovering over a topic,
topic cluster, or word) do not appear to be available yet in Vega~\cite{vega},
although the required components for reactive interaction were
recently described by Satyanarayan et
al.~\cite{Satyanarayan:2014:DID}.
Although custom Javascript was required, the flexibility is welcomed
by many web developers.

LDAVis highlights some unique features of RCloud.
While RCloud notebooks allow deployment of R analyses over
the web with no additional effort, RCloud \emph{applications} are more
powerful, and are developed with a combination of Javascript and HTML
for the front end. This requires additional knowledge over Shiny, but
the RCloud model makes the analysis side simpler (since analysts
simply write R in the style they already know)
and front end visualization side is simpler for the web
developers (since they simply write the Javascript code in
the style \emph{they} are used to). In addition, RCloud applications
inherit the automatic deployment and discoverability features of
all RCloud notebooks.

% IMPORTANT: what is unique about RCloud here
% From prototyping to dashboard
% Getting other information from the web
%
