\section{Case Study\label{sec:casestudy}: topic visualization in RCloud}

In this section, we present a simple, but real-world example
application currently deployed using RCloud.

This example application is an implementation of LDAVis, a
recently-published method for visualizing large collections of
text. It was specifically designed to enable non-expert users to
explore collections of short text documents using topic modeling and
visualization. Topic modeling~\cite{Blei:2003:LDA} is a standard
technique for text summarization which, although powerful, requires a
certain amount of manual intervention and
interpretation~\cite{Siever:2014:LAM}.

\begin{figure*}
  \includegraphics[width=\linewidth]{fig/casestudytext/casestudytext.pdf}
  \caption{\label{fig:textvis}An example application developed and deployed in RCloud.}
\end{figure*}

LDAVis itself was developed by two members of the technical staff at
AT\&T Labs, and originally was developed entirely in RStudio's
Shiny~\cite{RStudio:2013:SWA}, a framework for developing R
applications for the web. The main reason for moving the application
to an RCloud development was precisely ease of deployment. Shiny
provides ease of development, but when the application is finished,
one needs to find a publicly available web server that can host the
Shiny application; user management needs to be handled, and so
on. Using RCloud, deployment happens transparently. This is a big win
in terms of efficiency.

Describe the application and the technologies. Runs code from an R
package and custom analysis code; files are uploaded from the web
client, results are provided in JSON format and visualized in the web
client using d3~\cite{Bostock:2011:DDD}.



%% Text analysis in

%% To illustrate some aspects of RCloud, we present an example notebook
%% for stock price analysis.\footnote{Unfortunately this report could not
%% include production notebooks containing proprietary information.}
%% The example, although simplified, shows key steps in the development
%% and deployment of an application.

%% \begin{figure}
%% \includegraphics[width=\linewidth]{fig/casestudy1/casestudy1.pdf}
%% \caption{\label{fig:stockvis}Iterations of a stock ticker
%%   visualization, based on an example by Hadley Wickham. A
%%   simple, static visualization of the closing price of a single stock
%%   is progressively developed into a configurable display suitable
%%   for a dashboard, then into an interactive visualization that
%%   compares the volatility and volume of two stocks, and finally
%%   into an API for data access by other RCloud notebooks. The notebooks
%%   in this example can all be loaded as web pages. When a notebook
%%   corresponding to a function call is displayed a web page,
%%   its associated documentation is displayed.}
%% \end{figure}

%% It shows a sequence of visualizations of the performance
%% of financial stocks over several years. The first visualization
%% in this example uses ggplot2 and was written by Hadley Wickham.
%% It reads data provided by Yahoo! Finance as a web service, and shows
%% the price for a single trading symbol.

%% The webpage that produces that visualization includes a link to
%% the underlying source code. From this link, a user can fork the notebook.
%% In the example,a configurable ticker is added, based on the URL of
%% the notebook. The change to the notebook is minor.

%% Next, the notebook author or a collaborator may decide to extend
%% the application to also provide convenient access to the data,
%% apart from its visualization.
%% This is done by creating a notebook that defines a function.
%% That notebook then becomes a \emph{subroutine} for other notebooks.
%% It can be invoked from interactive notebooks, such as dashboards.
%% The data access notebook is version-controlled like other notebooks.

%% Finally, consider the situation where an analyst wants to understand
%% the price dynamics of stocks with respects to other attributes and
%% time ranges. For this, interactive visualization may be very helpful.
%% In our example, the analyst creates an interactive tool with multiple
%% linked views.

%% Demonstrating the integration of R and JavaScript, here the analyst
%% writes an R function that passes the dataframe and description of
%% the charts to the dcplot library.  Simple R expressions
%% are captured as trees to generate JavaScript expressions.
%% The terse chart description language, with sensible defaults inspired by
%% ggplot2, provides a simple yet powerful interface to the grouping
%% and reduction functionality of the well-accepted charting libraries
%% crossfilter, dc.js and d3.js.

% \subsection{Text analysis\label{sec:textvis}}
% Same.

% IMPORTANT: what is unique about RCloud here
% From prototyping to dashboard
% Getting other information from the web
%
