\section{Case Studies}

We present two example applications of the capabilities provided by RCloud
based on synthetic data and applications.\footnote{Unfortunately, we could
not show production notebooks because of concerns about revealing proprietary
or confidential information.} The main pont of the examples is that they are
representative of a progression through the visual analytics development
and deployment process that we aim to support.

\subsection{Stock Price Analysis\label{sec:stockvis}}

\begin{figure}
\includegraphics[width=\linewidth]{fig/casestudy1/casestudy1.pdf}
\caption{\label{sec:stockvis}Iterations of a stock ticker
  visualization, after an example by Hadley Wickham.  A
  simple, static visualization of the closing price of a single ticker
  is progressively developed into a configurable display suitable
  for dashboarding, then into an interactive visualization of the
  volatility and volume of two stocks, then finally into an API
  for data access by other RCloud notebooks. All notebooks
  in this example can be loaded as web pages. When a notebook
  corresponding to a function call is displayed a web page,
  its associated documentation is displayed.}
\end{figure}

Our first example is a sequence of visualizations of the performance
of financial stocks over a multi-year period. The first visualization
in this example uses ggplot2 and is due to Hadley Wickham. It reads
data provided by Yahoo! Finance as a web service, and shows the price
for a single trading symbol.

The webpage that produces that visualization includes a link back to
the generating source code, from which a user can fork the notebook
and, in this case, add a configurable ticker based on the URL of
the notebook. Notice how the notebook changes very little.

From there, the notebook author (or another user) may decide to
provide convenient access to the data apart from its visualization.
This is achieved by creating a notebook that defines a function.
This notebook then becomes a \emph{subroutine} for other notebooks,
and is version-controlled like other notebooks.

\todo{Interactive version.}

%% This notebook can then be called in an \emph{interactive}
%% visualization, which uses the Javascript 

%% Easy to convert it into configurable entry point for dashboarding.

%% Now consider an analyst that wants to understand the price dynamics of
%% different stock attributes, different points in time, etc. For that,
%% interactive visualization is very attractive. describe dcplot and
%% notebook which uses it.

%% Then describe abstraction of stock-data fetching, to talk about
%% calling notebooks from other notebooks.

%% then call.r?

\subsection{text analysis\ref{sec:textvis}}

Same.

% IMPORTANT: what is unique about RCloud here
% From prototyping to dashboard
% Getting other information from the web
%
