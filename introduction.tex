
\firstsection{Introduction}

\maketitle

\todo{Points we want to make."Coordination is done by meetings" "how do you trace an automated alert back to an EDA environment ?"}

% What's the area
Consider the emerging role of a data science team within an
organization today \cite{Keim:2008:VAS}.

% How do people do it today?
Individual data scientists and statisticians usually work on loosely
related problems, and must devise effective ways to share their findings
and move results from exploratory data analysis to automated diagnostics
and reports that are deployed for wider consumption.
They generally rely on a patchwork of resources shared informally,
or at best, through systems that attack individual tasks or functions
in the data analysis workcycle.

% Why does this not work?
There are two problems with the current
practice. First, there are gaps in the workflow:
Exploratory Data Analysis (EDA) is done with one set of tools,
and automated reports and deployments with another. This makes
transferring results from EDA to production slow, expensive, and error-prone.
Second, EDA environments often assume a single-developer perspective,
while data scientist teams could greatly benefit from convenient sharing of
scripts, data sets, data feeds, experiments, annotations, and automated
analyses and recommendations. These features are well beyond what existing
repositories and version control systems provide.

% What are we going to do about this?
\todo{Sell idea, not system} In this paper, we investigate the hypothesis
that the practice of visual analytics can advance by adopting techniques
from information retrieval and collaborative work environments. To investigate
these ideas, we designed and deployed a prototype environment called RCloud,
that supports collaborative data analysis, visualization and web deployment.
We will report on the design decisions, tradeoffs and limitations, and compare
the concepts in RCloud to similar proposals.

\todo{Define the requirements for the system}

We begin by noting the main requirements for a data analysis environment for
collaboration and transfer of results into production.

\begin{itemize}
\item Technology transfer. Experiments in notebooks developed by analysts need to become accessible in the production environment, making it easy to push new features, prototypes, etc.
\item Dashboarding. To support the above, there should be a mode in which non-experts
can operate the experiments without coding.
\item Version control as a primary concept in the repository. There needs to be a way to control the visibility of prototypes as they are being developed and tested.
\item Modern interactive technologies. People want to interact with data to explore it. Scalable technologies like crossfilter exist. Want to support those in the context of an R environment. Need to support full 2-way communication between computing in the cloud and interaction on the web.
\item Other requirements e.g. Kandel's user study. Not our contribution but things we must address, mentioned in related work section. Heer/Agrawala design considerations for collaborative visual analytics.
\end{itemize}

% What is the scope of our solution?
Our investigation focuses on the R language \cite{RCoreTeam:2013:R}
for statistical computing and graphics. R is one of the most popular
systems for ...

% We want our solution to play well with the rest of the ecosystem
R has developed a complex ecosystem of compatible packages, software
tools and other resources. For example there are packages that make
it convenient to execute map-reduce programs, generate web interfaces,

