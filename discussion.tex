\section{Lessons Learned, Discussion and Limitations}

What are the decisions and ideas that are central to this paper?

% limitations
Previous studies have pointed out difficulties in achieving the flexibility,
scalability and maintainability expected of production software in experimental
code written by data analysts.
Although these properties cannot be enforced by any programming environment,
suitable tools can help well-motivated analysts and programmers to create
robust applications with much less effort.

Asynchronous collaborative visual analytics
(ACVA)~\cite{Chen:2011:SEC}. This paper addresses visualization
\emph{of} the ACVA process. It will be important when we talk about
recommendation systems, and navigation of the set of notebooks, etc.

Had to back off only one language sooner than we expected.

Returning to the design considerations outlined by Heer and Agrawala, experience with the RCloud prototype demonstrates the value of several
Shared artifacts, artifact histories,
Discussion
View sharing - bookmarking, everything available as a URL
Content export - want to avoid, create ways of working where this is not necessary.
Social-psychological incentives - starring
Voting and ranking - starring

Group management, size, diversity.  Lacking support,but clearly desirable.

Has to play well in the ecosystem of the other tools and worked out as planned but there is a temptation to absorb

discussion about binding times, running vs.caching.
transparent vs. explicit operations done by programmers

What is relationship to a standard CMS with R/Python

Elaborate on to what extent can the users of the production notebooks use the tools like annotation etc. without coding or without being exposed to the analysts view. is this a gap in our design.

Shared Artifacts
Artifact Histories
Activity Histories


Discussion
Recommendation
Social-Psychological Incentives
Voting and Ranking


View Sharing / Bookmarking
Content Export
Presentation
