\section{Discussion, Lessons and Limitations}

\subsection{Design Considerations}

Experience with deploying RCloud in a community of data analysts
has provided some insight about how the proposed requirements
were met, and to what extent they were appropriate.
This experience may be informative to researchers and designers
of future systems supporting collaborative visual analytics.

Returning to the design considerations defined by Heer and Agrawala,
our experience underscores the importance of many of their proposals,
and indicates areas for further exploration.

{\bf Shared artifacts and artifact histories} are a central feature
of RCloud. The prototype aimed at making it convenient to collaborate
on experimental workbooks and the analyses that result. We observed
that analysts readily shared ideas through RCloud. They used RCloud
to discuss data analysis, programming problems, and to publish results
by sharing URLs instead of instead of embedding screenshots in
PowerPoint.\todo{Talk about artifact histories?}

{\bf View sharing, bookmarking.} Most resources in RCloud are named
and accessed as URLs. This proved to be an effective mechanism
for sharing, and for analyses to be integrated with
external processes and systems.

{\bf Discussion.} Annotation is supported in our human interface,
and was another central goal of our prototype. \todo{Elaborate.}
An interesting question is, to what extent should users of production
notebooks also able to use annotation and similar tools without coding
and without being exposed to the analyst's view?\todo{Is this a gap in
our design?}

{\bf Content export} is not a capability we worked to support,
but the R environment already has many packages for this.
Our goal was to devise ways of working where explicit export
is less necessary. However it is desirable to integrate results
through web protocols such as JSON. We did assume that RCloud
notebooks need to play well in the ecosystem of other tools,
and this worked out as intended, though we learned it also creates
a temptation to absorb features almost without limit. (Should
RCloud support templates for personal web pages and graphical
editing? Should it incorporate an email client and emacs?
Should it be able to push results to Facebook?)

{\bf Social-psychological incentives} and {\bf voting and ranking}
are supported through starring. This mechanism was employed often
by our test audience. An obvious next step would be to enhance
recommendations via relationships discovered by static and
dynamic code analysis. This should be done both within and across
collections of scripts (the latter being similar to Vistrails'
enhanced recommendations by clustering multiple workflows).
It should be valuable to data analysts to know which packages are
frequently used together, or appear with a given record type
or data feed. In general, though, trivial or passive mechanisms
to collect data for recommendations are essential.

{\bf Group management, size and diversity.} This area lacks sufficient
support in RCloud, but is clearly important to working in teams.
We rely on existing administrative and social processes and conventions,
outside RCloud, to manage accounts and groups. This capability
could likely benefit from integrating social media to track
identities and groups and to maintain communication channels.

\subsection{Limitations}

Because we developed the ideas and the prototype together,
we did not foresee some of the requirements that emerged
after our audience started working with the prototype.

We did not anticipate having to back off so soon on the
assumption that programming is being done mainly in one
language (R). In retrospect, it should have been obvious
that many data science problems are already being solved
with cross-language approaches.

A disadvantage is that some important aspects of the environment
are more difficult to manage in RCloud than in conventional systems.
Not only does it lack an explicit separation of the development and
deployment environments, but every version of every prototype is
shared by default. While this encourages collaboration on work
in progress, errors such as misconfiguration, mismatches between
versions of packages, and ordinary programming errors are also
easily shared and can affect production websites unintentionally.
This cannot be avoided entirely, but the power of sharing is worth
this price (as proven true with distributed systems in general).
Still, this problem needs more attention if the environment is
going to be scaled up to a wider number of users with fewer
informal channels to coordinate work. \todo{need a transition}
Versioning occurs in several places: in scripts, in the R environment
such as the installed R libraries and packages, and in the external
environment such as the operating system and even protocols spoken by
remote services. At one end of the scale, we have full control over the
versioning of scripts via git, though there is still a need for conventions
for example to name stable or working versions of scripts or versions known
to work together. The R environment is under the control of its package
manager. It is at least possible for RCloud to access information about
package versions and configurations, but support for checking compatibility
and maintaining multiple versions in the same environment is not strong.
At the other end of the scale, there is not much reason to expect version
control for the external environment. Most programmers rely on clean living
and a conservative approach to upgrades.

On top of this, RCloud allows analysts to fork experiments to try
new ideas quickly. But we have not done much so far to address the
problem of having a lot of bits and pieces of code and data lying
around, though at least we have made it easier to get at them and
to apply algorithms and metadata to organize them.

Asynchronous collaborative visual analytics
(ACVA)~\cite{Chen:2011:SEC}. This paper addresses visualization
\emph{of} the ACVA process. It will be important when we talk about
recommendation systems, and navigation of the set of notebooks, etc.

Discussion about binding times, running vs.caching.

Transparent vs. explicit operations done by programmers

Our environment assumes a high degree of trust between users.
Access control for information security is delegated to the host
operating system and web server. In practice, most operating
systems and web servers rely either on coarse permission models,
which tend to be ineffective, or on detailed access control
lists, which tend to be cumbersome and complicated. 
Information security in RCloud should be re-examined, but more
sophisticated approaches such as information flow analysis are
still an active research area, and many subtle problems remain
\cite{Moore:2011:SAF}. We are hopeful that having a richer
environment for collaboration and information sharing will encourage
practical applications for new approaches to information security
in visual analytics.
