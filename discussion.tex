\section{Discussion, Lessons and Limitations}
\label{sec:discussion}

\begin{table}
  \centering
  \begin{tabular}{l|cccccc}
    & V/F & C & D & ML & R & A \\
    \hline
RCloud     &       x            &       x       &     x      &      x        &           x        &           x         \\
RStudio    &                    &               &            &               &           x        &           x         \\
JSFiddle   &       x            &       x       &            &               &                    &                     \\
bl.ocks    &       x            &       x       &            &               &           x        &                     \\
shiny      &                    &               &     x      &               &           x        &           x         \\
Jupyter    &                    &               &            &      x        &           x        &           x         \\
Tableau    &                    &      x        &     x      &               &           x        &                     \\
ManyEyes   &                    &      x        &     x      &               &                    &                     
  \end{tabular}
  \caption{Summary of features for similar systems. The columns stand for, respectively, Versioning/Forking, Collaboration, Deployment, Multi-language support, Integrated Reports, and Integrated Analysis.}
\end{table}

\subsection{Reflection on Design Considerations}

Experience with deploying RCloud in a community of data analysts
(``hackers'') gave us some insight into whether the proposed
requirements were appropriate, and were met.
%
Our experience underscored the relevance of Heer and Agrawala's design
considerations, and indicated areas for further exploration~\cite{Heer:2008:DCF}. We adopt
their taxonomy in the following discussion.

\paragraph*{Shared artifacts and artifact histories} are a central feature
of RCloud, through shared workbooks. We observed that hackers readily
share experiments, demonstrate techniques, publish results to peers and
managers, and transfer algorithms to other groups. Artifact histories
can be accessed through the notebook tree or through GitHub's web interface.
But it is easy to create a large number of artifacts
and the system does not help enough to organize and navigate them.

\paragraph*{Modularity and granularity} Modularity, or the ability
to partition work into independent units, is a key to working
productively in teams.
If the units can be kept small or granular, team members
can realize benefits at least proportional to their work on the units.
RCloud's notebook and versioning allow work to be divided
into units as fine as the underlying language allows, and
encourage making incremental changes at low cost and without disrupting
the work of others.

\paragraph*{View sharing, bookmarking} Most resources in RCloud are named
and accessed as URLs. This proved to be effective for sharing analyses
and integrating them with external processes.  It is particularly
advantageous for work to be shared as URLs that provide access to live
notebooks, instead of by pasting screenshots into reports.

\paragraph*{Discussion} Annotation and commenting was another central goal.
Commenting is supported through GitHub, but our hackers found it
awkward and did not exercise it as much as we expected.
An interesting question is, to what extent should application users
be able to make annotations in published notebooks without coding
and being exposed to the hackers's view? The design of
more elaborate, integrated annotation remains as future work.

\paragraph*{Content export} is not a capability we aimed at supporting,
and R already has many packages for this. Recently, due to popular
demand, we added user interface support for exporting plot images.
RCloud notebooks should play well in the R ecosystem,
but sometimes it is difficult to know whether adding
a feature for compatibility will offload complexity, or increase it.

\paragraph*{Social-psychological incentives} and {\bf voting and ranking} are
supported through starring and forking. These mechanisms are employed often
on the platform. An obvious next step would be to enhance recommendations
using relationships discovered by static and dynamic code analysis. This
may be considered both within and across collections of scripts (the latter
being similar to VisTrails' enhanced recommendations by clustering multiple
workflows). It seems helpful to help users find which packages are frequently
used together, or when using record types or data feeds.
Overall, simple, passive mechanisms to collect data for recommendations are preferred.

\paragraph*{Group management, size and diversity} This area needs better
support, and is clearly important to working in teams.
We rely on external administrative processes and social conventions
to manage accounts and groups. RCloud could benefit greatly from
integrating social media to track identities and groups and to maintain
communication channels, instead of having its own isolated solution.

\paragraph*{Curation} Even without formal group management, users often
organize groups of related notebooks using tree folders.
We found this particularly effective in collecting and distributing
training materials.

\subsection{Limitations}
Because we developed our ideas while simultaneously creating a prototype,
we did not foresee some of the requirements that emerged after
people started working with RCloud.

\paragraph*{Versioning and Upgrading}
Some aspects of the environment are more difficult to
manage in RCloud than in conventional systems. RCloud does not separate
its development and deployment environments, and every version of every
notebook is shared by default. Although this encourages
collaboration, it also quickly exposes misconfigurations,
mismatches between package versions, and programming errors
that can unintentionally affect production websites.
%% If an RCloud website is distributed across multiple servers,
%% conflicting versions of system software
%% can also cause errors that are hard to debug.
Versioning of software components must be managed in several levels: in
the notebooks themselves, in the installed libraries and packages of the
R environment, and in the external environment such as the operating
system, libraries, and protocols spoken by remote services.
%% At one end of the scale, we have full control over notebooks
%% via git, along with conventions to fix stable notebook versions.
%% The R environment is under the control of its package manager that
%% has rules to ensure reasonable consistency.
%% % (It is possible for RCloud to access information about package
%% % versions and configurations, but support for checking compatibility
%% % and maintaining multiple versions in the same environment is not
%% % strong.)
%% At the other end of the scale, there is basically no
%% control over the external environment, and if an external service 
%% used by a notebook changes its API, the notebook can stop working.

%Asynchronous collaborative visual analytics
%(ACVA)~\cite{Chen:2011:SEC}. This paper addresses visualization
%\emph{of} the ACVA process. It will be important when we talk about
%recommendation systems, and navigation of the set of notebooks, etc.

\paragraph*{Security}
Because RCloud allows arbitrary R code to be run
\emph{by design}, it's not straightforward to protect deployments
from running unauthorized arbitrary code. RCloud uses an object capability model
\cite{Miller:2006:RCT} recently added to the Rserve
protocol~\cite{Urbanek:2003:AFW}. Object capabilities are opaque handles
to remote procedure calls: web browsers never directly instruct the
RCloud backend to execute arbitrary code, preventing unauthenticated clients
from making unauthorized calls to the RCloud runtime environment.
Our back-end environment, on the other hand, assumes a high degree of trust between users:
access control is delegated to the host operating system and web server.
Most operating systems rely either on coarse permission models,
which tend to be ineffective, or on detailed access control
lists, which tend to be cumbersome.
Information security in RCloud today is an unsolved matter.
More sophisticated approaches are open research topics~\cite{Moore:2011:SAF}.

\paragraph*{Nonforgetful, effortless exploration is hard}
Users reported that RCloud's interface clashes with the
typical use of R for exploration.
In current R practice, the command prompt is used to try things out.
It is very fast and there is often no commitment: nothing gets saved by
default, and clean up is automatic. In contrast, in RCloud,
cells generated during exploration are saved in notebooks.
We arrived at this from an insistence on reproducibility: 
no command should be run without being saved.
In our experience, some history steps are only found to be
important in hindsight, and deleting history by default makes
it too easy to lose those steps by accident.

We found that when users realize that \emph{scratch work
is getting saved}, its sheer volume becomes burdensome.
Clearly, RCloud is an extreme, and the right solution is
probably in the middle. An intriguing possibility,
suggested by one of our interviewees, is to implement auditing of data
analyses as it existed in S~\cite{Becker:1988:Auditing}, tying
data artifacts to the processes that generated them. 
Bookkeeping would be reduced to selecting the \emph{data} to be kept,
and the code that generated it would be derived and saved transparently.

\paragraph*{Discovery is hard} 
With hundreds of users and thousands of notebooks, a global notebook tree
becomes unwieldy. The search function does not protect searchers from
obsolete and erroneous notebooks, which by default are retained
forever. In addition, important metadata (has this notebook been
recently run, or edited?) is not easily searchable. In other words,
\emph{textual search} is easy, but \emph{relevance search} is what we
need. Allowing users to \emph{tag} notebooks is natural, but
relies on users making a conscious effort, which goes against
our ideal of not asking the user to do work for the system.
Automatic
\emph{curation} of
software artifacts generated by teams seems to be an important avenue
for future work.

\paragraph*{Collaboration is hard}
While forking provides a simple way to continue someone else's work,
teams working together generate workflows for which forking is not
ideal. Although popular, forking poses a problem when dozens of
versions of a notebook proliferate. A fully general solution seems
intractable. Nevertheless, we expect future versions of the
notebook infrastructure to move toward git's ideal of decentralized
code repositories, instead of the exclusive ownership mode RCloud has now.
