\section{Conclusions and Future Work}
We designed, implemented and deployed a prototype visual analytics environment
for data exploration, sharing, presentation, and publishing. It is a step toward
practical ``DevOps for data science'' and reproducible, publishable data
science experiments.

RCloud was readily accepted as a platform for deployment of
visualizations and interactive exploration tools inside our organization.  
Notebook sharing and publishing were eagerly adopted.
Features for single-user data exploration, on the other hand, that
compete with existing mature tools, were not readily accepted.
In the future we hope to study and understand the degree to which
this reluctance is caused by bad design decisions on our part, and how
much comes from ``mere'' change aversion. It is also clear we need
need better ways to manage the persistence of notebooks and to cope
with potential information overload.

We find much evidence that the biggest barrier to the adoption of
visual analytics is inadequate software infrastructure.
Just as structuring visualization software around simple,
composable parts is highly successful (exemplified by
Grammar of Graphics approaches like Vega and ggplot)
it is intriguing to consider what systems will be
possible when this philosophy is extended to the
broader requirements of interactive analysis.

RCloud is open-source, and available at \url{github.com/att/rcloud/}.
