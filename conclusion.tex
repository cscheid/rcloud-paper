\section{Conclusions and Future Work}

\stephen{Is this the kind of bad conclusion that contains
a lot of material that should have been explained at the
beginning of the paper to move it along faster and tell
people why they should be interested in the work? What
would be our conclusion?  These capabilties are important,
but they are really hard to program, and adoption is slow
because you're meddling with people's work process and
toolset, which is the heart of their environment and
not some little toy you provide out on the fringes.
Infratsructure is hard.}

We presented a case for an environment that supports the
visual analytics process for work by teams. The process
includes data acquisition, exploratory data analysis,
code development and deployment.

Building on previous work to define requirements for
visual analytics, we designed a prototype environment, RCloud.
RCloud supports collaboration and communication
through shared experimental workbooks with version
control, searching, recommending, annotating, and publishing
experiments as web sites and as reusable services. 

We also deployed RCloud in a community of working data scientists.
Experience with the RCloud prototype provides evidence that data
science teams and the organizations in which they work benefit from
having capabilities that support collaboration and integrate
the entire visual analytics process.

RCloud is a step toward practical ``DevOps for data science'' and
reproducible, publishable experiments.

Possible next steps are to support cross-language development,
to incorporate richer recommendation techniques, to provide fine-grained
information security, and to improve the usability of the human interface.

RCloud code is available at \url{github.com/att/rcloud/}
under an MIT open source license.
