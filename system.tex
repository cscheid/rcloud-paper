\section{The System\label{sec:system}}

\subsection{High-level Architecture\label{sec:highlevelarchitecture}}

% Stephen: Let's work from front to back, from human interface to the
% services and frameworks that support it.

The internal computing infrastructure of organizations has changed
radically in the last fifteen years. The shift from large
servers toward scalable, lower-cost, distributed systems (``the cloud'')
led to a software ecosystem of processes distributed over a
network, usually communicating via HTTP. HTTP is dominant because
web browsers and servers are ubiquitous and available in almost
all hardware devices, from tiny sensors, to handheld devices and
laptops, to rack-mounted servers.

As a result, HTTP is the lingua franca of interprocess communication
(IPC). One of the design goals for RCloud was to provide a productive
environment for creators of data-analysis scripts, that also behaves
as a first-class citizen in the pre-existing ecosystem of computer
services and networks in an organization.

%% Design for cloud-friendly? This goes back to the point in the
%% introduction about playing nice with the rest of the ecosystem.

\subsection{Human Interface\label{sec:humaninterface}}

%\stephen{What??}
%We designed RCloud around the front-facing API, which provides roughly
%one entry point to correspond with each requirement.

Figure~\ref{fig:notebook} shows the RCloud developer interface.  The center
panel is a notebook for code and visualizations.  Code is edited in this panel,
and visualizations are rendered.  An inventory of supplementary ``asset'' files,
that are part of the notebook, though not in its executable flow, are edited on
the right.  Controls at the top of the screen allow the user to run, view, and
share the notebook. \gordon{Why are we talking about the callable interface
here?} The result of code execution is the final cell of the notebook.  On the
left is a browsing area for searching and viewing other users' notebooks, and a
help system.

The call mechanism is the URL itself; arguments are read from the URL
and the final cell is rendered as the result.

\subsection{Notebooks\label{sec:notebooks}}

The main unit of computation in RCloud is a \emph{notebook}.
A notebook holds a sequence of \emph{cells}, each of which contains a
snippet of code or hypertext in Markdown. This is not a novel idea;
executable documents structured this way are a feature of many
other systems, including Mathematica, IPython and Sage.
\stephen{Code is executed when...?}

One of the main contributions of RCloud is the idea that notebooks
are ``always deployed''. In other words, the most recent version of
a notebook is immediately available.\stephen{What if it is broken?}
This makes it convenient to share and modify experiments and compose
results without binding to a specific version of a notebook.
On the other hand, there are situations where it might be important
to name specific versions. We do not expect designers to decide which
versions need to be preserved, but we embrace \emph{transparent} versioning.
This is similar to models like Jankun-Kelly et al.'s p-set calculus \cite{Jankun-Kelly:2007:MFV}
and VisTrails's version tree \cite{Callahan:2006:VVM}, where every change in the state of the system is tracked.

To implement this, we built RCloud on top of GitHub's \emph{gists}~\cite{GitHub:2014:GG}.
GitHub offers a HTTP interface for creating simplified git repositories, the main limitation
being a restriction to text-only files in a single directory. The GitHub web-service
API provides most of the semantics we need for the versioning portion of the storage back end:
access to previous versions, comments, starring, and forking.

Using GitHub for storage and versioning also exposes other capabilities
that can be invoked with JSON and HTTP.
Particularly, we can provide full text search using Apache SOLR.
SOLR is scalable, can index in near-realtime, supports multiple character sets,
indexes several common types of documents, and has schemas and faceted search.
%
Integrating existing services and software components, especially open source,
instead of implementing custom code is a trend in the software industry, and
is very positive for systems research and prototyping in visual analytics.
By adopting existing technology, small teams with limited resoures can explore
novel ideas toward the improvement of large, complex software ecosystems.
In our case, even though this approach involved learning standards not directly
related to visual analytics, our early decision to use the GitHub Gist API
turned out to be successful. Future projects could gain many of the same benefits
by adopting this strategy. In fact, it will be essential for technology adoption
and transfer, because it is almost impossible for any one tool or platform to
``own'' or support the entire visual analytics process.
\stephen{Are we preaching too much?}

\subsection{Reputation and Interest: starring\label{sec:starring}}

Information retrieval based on collecting usage and recommendations
is a cornerstone of modern web services. We would like to help data
scientists to find workbooks (and therefore items in their contents
such as code, data, and colleagues with specific expertise) with
the benefit of such information.

In RCloud, reputation and interest are a relationship between
\emph{notebooks} and \emph{users}, rather than a relationship between
user pairs. We chose this approach because we expect initial 
RCloud deployments to have relatively few users, but some users to
create many notebooks. Under that assumption, assigning interest
to users would not provide sufficiently ``high-resolution'' data.

We incorporate both explicit and implicit indications of interest
in notebooks. Explicit interest is indicated by ``starring,'' or
clicking on a button that marks a notebook as interesting.
This makes explicit indication of interest a nearly trivial operation,
always readily available, to encourage its use.

Implicit signaling of interest is supported by keeping click-through counts
\cite{Joachims:2005:AIC} and execution counts. (In addition to these
standard techniques of collecting feedback from web search, we anticipate
applying static and dynamic code analysis to infer fine-grained
information about relationships, for example, which packages and data
sets often appear together.)

\subsection{Executing R in a web browser\label{sec:Rinbrowser}}

One our main goals is to provide ubiquitous access to the statistical
tools of the R programming language in the wider environment in which data
science teams participate.
%
Given the recent developments of interactive visualization and visual
analytics in HTML5 and the web, we decided to create an
\emph{interconnect} between the two environments.
%
Every RCloud session spawns a new R execution process in a remote
server.

%% \subsubsection{Object capabilities: R to Javascript RPC and access
%%   control}

HTTP and web services are convenient but much of the protocol is
stateless. (For example, GET requests are required to not change the
remote state, and results can be cached in transparent proxies along
the network). In the case of close communication between a running
R process and a web browser, we need something that is more flexible
than distinct URLs for every R process, and which embraces
statefulness of both calculations and visualization parameters.

\carlos{In-depth explanation of R<->web interconnect, and why it's
  necessary or important. Main point: make programming in the R side
  as close as possible to what R programming feels like, and make
  programming in the Javascript side as close as possible to what
  Javascript programming feels like. We do this because this part of
  the system is targeted at programmers; Shiny targets web programming
  at non-web-programmers. We don't attempt such a thing and land at a
  different point in the spectrum.}

\carlos{compare against https://www.opencpu.org, compare against
  IPython notebooks. Two-way communication? Actually describe what it
  is. IPython manipulate notebooks? Understand and contrast}

\subsection{Interactive notebooks\label{sec:interactivenotebooks}}

%% How do we do things that are not trivial to do with IPython (for
%% example)

%% What is relationship to a standard CMS with R/Python

%% dcplot. two-way communication between between backend session and
%% frontend session.

A key engineering decision in RCloud was to rely on full two-way
communication between the web client human interface, and data and
computing resources in the cloud. It is easier to engineer a design
in which the human interface is pushed one-way from the server
and thereafter all interaction takes place within the client.
Two-way communication is necessary, though, where the size of
the data or computational performance makes it impractical to
move all computation to the client.

In addition, R is well suited as a language for analysis, and
JavaScript for interactive visualization. To draw on
the strengths of both languages and both environments, the connection
between the languages is not just procedural: it makes \emph{closures}
and \emph{first-class functions} available across the network.
This provides considerable flexibility, so that for example, a chart built
with dc.js or leaflet.js can call back to analysis functions in R
without having to formalize the protocol between the processes.

\subsection{Deployment of notebooks\label{sec:deployment}}

Every notebook in RCloud is named by a URL, and notebooks by
default are visible by the entire organization. This is deliberate.
As pointed out by Wattenberg and Kriss~\cite{Wattenberg:2011:DFS},
broad access to analysis outputs (in their case, for NameVoyager)
increases long-term engagement in part through cross-references on
the web. Although our prototype RCloud deployment is only visible
inside a corporate intranet, we nevertheless found anecdotal support
for this notion by discovering links to RCloud notebooks in internal
discussion fora and mailing lists.

Because notebooks are continually published, any work is immediately
available both as a subroutine and a visual component. Close colleagues
can start on the next stage of analysis, or delve into the data,
even while the original author is polishing an algorithm or its
presentation. \stephen{We glossed over the things that can go wrong.
We should at least discuss that briefly.}
The code is the page, and can either be shaped into
a function of inputs and outputs, or the linear cells of the notebook
can be reworked into a full-fledged HTML layout.
