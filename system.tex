\section{The System\label{sec:system}}

Our design takes advantage of existing web software as much as possible.
HTTP, despite its deficiencies, is the lingua franca of distributed
interprocess communication. By speaking HTTP natively, our system
provides a low-friction path from experimental data analysis scripts
to automated \emph{web services}: essentially a remote function call over
the web which can be invoked by higher-level tools.
%An example of
%this sort of progression is described in Section~\ref{sec:casestudy}.

In consequence, the entire high-level infrastructure of RCloud, shown
in Figure~\ref{sec:system}, uses web standards. The communication between
a web browser and an active R session as a user edits a notebook is
performed by a combination of HTTP and Websockets, while all other IPC
is done through HTTP, from notebook versioning in GitHub to building and
maintaining full-text indices through SOLR. The most novel aspect
of RCloud's runtime system is its tight integration between the web client
and the backend R process, described in Section~\ref{sec:Rinbrowser}.

\subsection{Notebooks\label{sec:notebooks}}

The main unit of computation in RCloud is a \emph{notebook}.  A
notebook holds a sequence of \emph{cells}, each of which is a snippet
of code or hypertext in Markdown. As mentioned in
Section~\ref{sec:related}, executable documents like
this are a feature of previous systems, including Mathematica,
IPython and Sage.

Notebooks can be executed one cell at a time in an interactive
session, similar to traditional read-eval-print
loops, or can all be executed concurrently,
similar to running a shell script.

One of the main contributions of RCloud is the notion that notebooks are
``always deployed''. In other words, the most recent version of a
notebook is immediately available to all other users of the system.
Another way to describe this is that
RCloud lacks a ``save'' button: any notebook cell that runs is always
associated to a notebook version serialized to disk. Although one of
our interview subjects reported that this sometimes leads to an
excessively fine-grained sequence of versions (see
Section~\ref{sec:interviews}), 
losing information that could be shared is worse (see
Section~\ref{sec:interviews} user comments on
satisfaction with low-friction shareability).

In this design, we get notebooks that are always live,
but sometimes broken. Because stability is important, we also
allow any previous version of a notebook to be tagged and referenced.
Our notebook scheme is similar in many ways to models like
Jankun-Kelly et al.'s p-set calculus
\cite{Jankun-Kelly:2007:MFV} and VisTrails's version tree
\cite{Callahan:2006:VVM}, where every change in the state of the
system is tracked.

RCloud's implementation of the versioning mechanism is built on
top of GitHub's \emph{gists}~\cite{GitHub:2014:GG}, which are an HTTP
interface for simplified repositories. The GitHub web-service API
provides most of the semantics we need for the versioning portion of
the storage back end: access to previous versions, comments, starring,
and forking. This provides the additional benefit that every RCloud
notebook can also be manipulated like a git repository, granting advanced users
access to features like command-line checkout and history
editing.

\subsection{Reputation and Interest: starring\label{sec:starring}}

A side benefit of centralizing the execution and storage of
notebooks is that it becomes feasible to collect usage information
that is lost in conventional environments.
In the case of social data visualization
platforms, we would like to exploit usage data to help analysts
find content of interest, whether that content is source code
or data. Standard ways to achieve this are through
\emph{user-generated curation} and \emph{automatic recommendations}.

Automatic recommendations have become famous in the user experience
provided by companies such as Amazon and Netflix (``If you liked
that film, you'll like this one too''). To make such
recommendations, we need users to \emph{curate} the collection, by
providing explicit reviews or some other method of indicating interest.
%%
%% In RCloud, reputation and interest are a relationship between
%% \emph{notebooks} and \emph{users}, rather than a relationship between
%% user pairs. We chose this approach because we expect initial 
%% RCloud deployments to have relatively few users, but some users to
%% create many notebooks. Under that assumption, assigning interest
%% to users would not provide sufficiently ``high-resolution'' data.
%%
We incorporate both explicit and implicit indications of interest
in notebooks. Explicit interest is indicated by ``starring,'' or
clicking on a button that marks a notebook as interesting.
This makes explicit indication of interest a nearly trivial operation,
always available and easy to use.
%
RCloud today uses only simple counts of stars to measure overall
notebook interest, mostly because standard
recommender system algorithms 
require reasonably large training sets to pay off.
Nevertheless, the starring mechanism is sufficient
to create personalized recommendations~\cite{Hu:2008:CFF}.

Implicit signaling of interest is supported by keeping click-through
counts~\cite{Joachims:2005:AIC} and execution counts. In addition to
these standard techniques, we hope to apply static and dynamic code
analysis to infer fine-grained information about relationships,
for example, which packages and data sets often appear together,
% in the style of Fast et al's Codex system.~\cite{Fast:2014:ECS}.
in the style of Codex~\cite{Fast:2014:ECS}.

\subsection{Deployment of notebooks\label{sec:deployment}}

Every notebook in RCloud is named by a URL, and notebooks by default
are visible in the entire organization. This is deliberate.
As pointed out by Wattenberg and Kriss~\cite{Wattenberg:2011:DFS},
broad access to analysis outputs (in their case, for NameVoyager) increases
long-term engagement in part through cross-references on the
web. Although our prototype RCloud deployment is only visible inside a
corporate intranet, we nevertheless found support for this notion by
discovering links to RCloud notebooks in internal discussion fora and
mailing lists. In addition, as we describe in
Section~\ref{sec:interviews}, users have almost unanimously adopted
``share-by-URL'' as their default communication mechanism, as opposed
to ``share-by-screenshot'', which we consider to be an encouraging
validation of the system.

\subsection{Executing R through a web browser, and Javascript through an R process\label{sec:Rinbrowser}}

As mentioned, the other main goal in RCloud was to provide
full access to the R statistical programming language during the
\emph{development} of a data analysis notebook.
At the same time, when notebooks are \emph{deployed} (and potentially
accessed by anyone with a web browser), we'd like to allow the
browser to invoke only a very limited subset of R, namely those
notebooks that have been published.

The solution we developed is simple and general, and was directly
inspired by Miller's \emph{object
  capabilities}~\cite{Miller:2006:RCT}. Specifically, the R layer that
communicates with Javascript does not expose unprotected evaluation of
arbitrary functions. Instead, every function that the R layer intends
to expose is associated with a large, cryptographically-safe random
number (a ``hash''). This random number is then sent
across the wire and interpreted as an opaque function identifier.
Because these identifiers are cryptographically safe, all
the Javascript layer can do is send them to the R side, in a
message requesting a function call. 
The result of this function call might include \emph{new}
opaque identifiers, exposing new ``capabilities'' to the client.

The same idea of exposing functionality via hashes can be used to give
the server-side of RCloud access to Javascript functions. This allows
R libraries to request user input in the browser, giving them
access to features ranging from password prompts, to the currently
selected set of points in an interactive visualization.

As a result, the features required to provide safe access to R from
the client, and Javascript access from the R side, also enable
full two-way communication between the languages.  This provides
considerable flexibility, so that for example, a chart built with dc.js
or leaflet.js can call analysis functions in R without having
to define an additional protocol between the processes. Calls in either
direction look like ordinary function calls.
%% gw: inaccurate, it's websocket; we still don't support HTTP into a live process
%% From the
%% Javascript side, an RPC call into the R process is just another HTTP
%% request. From the R process, a call into Javascript looks like just
%% another function call.
