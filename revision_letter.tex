\documentclass{article}
\usepackage{microtype}
\usepackage{color}
\title{Letter to Reviewers of Submission \#341}
\begin{document}
\maketitle

We thank the reviewers for their careful reading of the manuscript and
their suggestions. In this document, we summarize the changes we
performed to the original submission.

Throughout the text, we have attempted to clarify that although RCloud
itself does not present a novel visual encoding (and isn't a pure
visual analysis system either), we still believe it presents an
important contribution to VAST. That's because we have found that, in
our experience in deploying visualization solutions in a larger
organization, \textbf{the infrastructure for deploying a large number of
  changing, exploratory visualizations is itself the bottleneck for
  visualization adoption}. Once we accept that visualization is a part
of a wider data analysis infrastructure in a larger organization,
\textbf{failure to tackle the larger issues will mean that visualization
solutions simply do not get adopted. We want to avoid this future}.
RCloud is an attempt to fix the problems we have found to be the most
pressing, namely of managing a large number of changing visual
analysis solutions deeply connected to statistical models and data
access layers, all while maintaining easy deployment and portability.

\section*{Specific changes, by review criteria}

We will answer the primary reviewer, point by point.

\paragraph*{1. Strengths and weaknesses}
We have kept most of the content of the interviews, but structured it
differently to provide more context, as suggested by a reviewer. On
the other hand, we have chosen to keep the details about the version
control and execution, because they are central to the system design
of RCloud, and they enable the collaboration modes we wish the system
support. As we've argued above, inadequate infrastructure for sharing
and searching for different analyses and visualizations is becoming
the bottleneck in large organizations, and we believe our
infrastructure offers a novel attack on the problem.

\paragraph*{2. Is the work novel, incremental, or previously published? 3. Is this work relevant for VAST?}
We accept that there are no novel visual encodings in RCloud; our goal
was to provide supporting infrastructure for visual and exploratory
analysis as a central tool for discovery in a larger organization. We
believe there are many previously published VAST papers falling into
this broader category, and would like to argue for our submission to
be considered similarly.

\paragraph*{4. Is related work adequately referenced?}
Even though the primary reviewer did not request any changes on this
point, later in the review (specifically in point 7) they mentioned a
missing comparison to ManyEyes in Table 1. We have added this
comparison.

\paragraph*{5. Are the technical results sound?}
We apologize for the lack of other examples, but we chose LDAVis as
one application to showcase RCloud because, to put it completely
frankly, we are not allowed to disclose the other applications being
developed at AT\&T Labs. LDAVis has been given publication clearance
\emph{and} is developed in RCloud. We understand that this is not a
completely satisfactory solution, but again, we believe the
fundamental contribution of RCloud is to reduce the barriers for
exploratory visual analysis in larger organizations, rather than any
specific visual encodings. Since RCloud is built on top of R and
HTML5, any technologies spanning this subspace are usable: LDAVis uses
D3, but one can similarly develop visual encodings using other
techniques such as WebGL.

\paragraph*{6. Is the exposition clear? 7. Should anything be deleted or condensed?}
We have reduced the length of the introduction, as requested, but kept
the system description to most of its current length. We believe that
exposition to be important for the completeness of the manuscript.

\paragraph*{8. Are the figures informative?}
We have extended the captions of the requested figures.

\section*{Other issues}

\paragraph*{Paper Length} We have reduced the length of the paper
to eight pages, removing only extended discussion in the limitation
and interviews sections. We achieved most of the space savings was
achieved by a redesign of the figures and tables, and rewriting the
text.

\paragraph*{Section Lengths} The reviewers were concerned about the length
of the introduction. We have significantly reduced it. Unfortunately,
space constraints meant that we could not add more details to the
system section.

\paragraph*{Should anything be deleted or condensed?}
We have added the comparison to ManyEyes in Table 1, as suggested by the review.

\paragraph*{Typos, missing references} We have corrected all missing references and typos.

\end{document}
