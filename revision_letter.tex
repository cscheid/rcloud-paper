\documentclass{article}
\usepackage{microtype}
\usepackage{color}
\title{Letter to Reviewers of Submission \#341}
\begin{document}
\maketitle

We thank the reviewers for their careful reading of the manuscript and
their suggestions. As requested, we here summarize the changes
made to the previous draft that was reviewed.

Throughout the text, we attempted to clarify that although RCloud
itself does not present a novel visual encoding (and isn't a pure
visual analysis system either), it presents a valuable contribution
to VAST. That's because we have found that, in
our experience in deploying visualization solutions in a larger
organization, \textbf{the infrastructure for deploying a large number of
changing, exploratory visualizations is itself the bottleneck for
visualization adoption}. Once we accept that visualization is a part
of a wider data analysis infrastructure in a larger organization,
\textbf{failure to tackle the larger issues will mean that visualization
solutions simply do not get adopted. We want to avoid this future}.
RCloud is an attempt to fix the problems we found to be the most
pressing, namely of managing a large number of changing visual
analysis solutions deeply connected to statistical models and data
access layers, all while maintaining easy deployment and portability.

\section*{Specific changes}

\paragraph*{Paper Length} We reduced our submission to eight pages,
mainly cutting the extended discussion in the limitation
and interviews sections. We achieved most of the space savings
by a redesign of the figures and tables, and tightening
the text.

\paragraph*{Section Lengths} The reviewers were concerned about the length
of the introduction. We significantly reduced it. Unfortunately,
space constraints meant that we could not add more details to the
system section.

\paragraph*{Comparisons} We added a comparison to ManyEyes to Table 1.

\paragraph*{Captions} We added appropriately-sized captions to Figures 2 and 4.

\paragraph*{Typos, missing references} We have corrected all missing references and typos.

\end{document}
